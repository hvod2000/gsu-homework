\setuppapersize[A4]
\setuplayout[backspace=2cm, topspace=1cm, header=1cm, bottomspace=1cm,
    footer=1cm, width=middle, height=middle]
\setupbodyfont[libertinus, 14pt]
\mainlanguage[ru]
\setuppagenumbering[location={footer,center}]

\setupformulas[align=right]
\starttext
\setuppagenumbering[state=stop]
\centerline { Гомельский государственный университет имени Франциска Скорины }
\centerline { факультет математики и технологий программирования }
\vfill \vfill
\centerline { Лабораторная работа №3 }
\centerline { «Целые числа» }
\vfill \vfill
{\leftskip 0.55\hsize \noindent
  Выполнил:\\Хамков Владислав (ПИ-11)\\
  Проверил:\\Васильев Александр Федорович}
\vfill
\centerline { Ноябрь 2021 }
\page

\setuppagenumbering[state=start]
\setuppagenumber[number=1]
\centerline {Задание №1}
\startformula \startalign[align=left]
\NC \forall a \in \mathbb{N}\ \ a = 3^{3n+2} + 5 \cdot 2^{3n+1},\ b = 19,\quad b\ |\ a? \NR
\NC a_1 = 3^{3+2} + 5 \cdot 2^{3+1} = 243 + 5 \cdot 16 = 323 = 17 \cdot 19
    \Rightarrow b\ |\ a_1 \NR
\stopalign \stopformula
Предположим верность утверждения для $n = k$ и докажем его для $n = k + 1$:
\startformula \startalign[align=left]
\NC a_{k+1} = 3^{3(k+1)+2} + 5 \cdot 2^{3(k+1)+1} = 3^{3k + 5} + 5 \cdot 2^{3k + 4}
    = 27 \cdot 3^{3k + 2} + 8 \cdot 5 \cdot 2^{3k + 1} = \NR
\NC = 19 \cdot 3^{3k + 2} + 8(3^{3k + 2} + 5 \cdot 2^{3k + 1})
    = 19 \cdot A + 8 (19 \cdot B) = 19 \cdot (A + 8B) \NR
\stopalign \stopformula
Таким образом $ b\ |\ a_n\ \ \forall n \in \mathbb{N}$ по индукции.
\vfil


\centerline {Задание №2}
\startformula \startalign[align=left]
\NC a = \pm751,\ b = \pm22,\ a / b =\ ? \NR
\NC +751 = (+34) \cdot (+22) + 3 \NR
\NC +751 = (-34) \cdot (-22) + 3 \NR
\NC -751 = (-35) \cdot (+22) + 19 \NR
\NC -751 = (+35) \cdot (-22) + 19 \NR
\stopalign \stopformula
\vfil


\centerline {Задание №3}
\startformula \startalign[align=left]
\NC a = -36248,\ b = -159,\ d,r =\ ? \NR
\NC -36248 = (-159) \cdot (228) + 4 \NR
\NC d = 228,\ r = 4 \NR
\stopalign \stopformula
\vfil


\centerline {Задание №4}
\startformula \startalign[align=left]
\NC a = 1716,\ b = 1540,\ НОК, НОД =\ ? \NR
\NC НОД(a, b) = НОД(1716, 1540) = НОД(1 \cdot 1540 + 176, 1540) = \NR
\NC = НОД(1540, 176) = НОД(8 \cdot 176 + 132, 176) = \NR
\NC = НОД(176, 132) = НОД(1 \cdot 132 + 44, 132) = \NR
\NC = НОД(132, 44) = НОД(3 \cdot 44, 44) = 44 \NR
\NC НОД(a, b) = 44 = 176 - 132 = 176 - (1540 - 176 * 8) = \NR
\NC = -1540 + 9\cdot 176 = -1540 + 9(1716 - 1540) = 9 \cdot 1716 - 10 \cdot 1540  = 9a - 10b\NR
\NC НОК(a, b) = \frac{a \cdot b}{НОД(a, b)} = \frac{2642640}{44} = 60060 \NR
\stopalign \stopformula
\page


\centerline {Задание №5}
\startformula \startalign[align=left]
\NC a = 82720,\ b = 63800,\ НОД =\ ? \NR
\NC НОД(a, b) = НОД(82720, 63800) = НОД(8 \cdot 10340, 8 \cdot 7975) = \NR
\NC = 8 \cdot НОД(10340, 7975) = 8 \cdot НОД(4 \cdot 2585, 7975) = \NR
\NC = 8 \cdot НОД(2585, 7975) = 8 \cdot НОД(2585, 7975 - 2585) = \NR
\NC = 8 \cdot НОД(2585, 5390) = 8 \cdot НОД(2585, 2 \cdot 2695) = \NR
\NC = 8 \cdot НОД(2585, 2695) = 8 \cdot НОД(2585, 2695 - 2585) = \NR
\NC = 8 \cdot НОД(2585, 110) = 8 \cdot НОД(2585, 2 \cdot 55) = \NR
\NC = 8 \cdot НОД(2585, 55) = 8 \cdot НОД(2585 - 55, 55) = \NR
\NC = 8 \cdot НОД(2530, 55) = 8 \cdot НОД(2 \cdot 1265, 55) = \NR
\NC = 8 \cdot НОД(1265, 55) = 8 \cdot НОД(1265 - 55, 55) = \NR
\NC = 8 \cdot НОД(1210, 55) = 8 \cdot НОД(2 \cdot 605, 55) = \NR
\NC = 8 \cdot НОД(605, 55) = 8 \cdot НОД(605 - 55, 55) = \NR
\NC = 8 \cdot НОД(550, 55) = 8 \cdot НОД(2 \cdot 275, 55) = \NR
\NC = 8 \cdot НОД(275, 55) = 8 \cdot НОД(275 - 55, 55) = \NR
\NC = 8 \cdot НОД(220, 55) = 8 \cdot НОД(4 \cdot 55, 55) = \NR
\NC = 8 \cdot НОД(55, 55) = 8 \cdot 55 = 440 \NR
\stopalign \stopformula
\vfil


\centerline {Задание №6}
\startformula \startalign[align=left]
\NC НОД(a,b) = 18, НОК(a,b) = 4896,\ a,b =\ ? \NR
\NC НОК(a,b) = 4896 = 2^5 \cdot 3^2 \cdot 17 = 18 \cdot 17 \cdot 2^4 \NR
\stopalign \stopformula
Ответы:
\startformula \startalign[align=left]
\NC a = 18 \cdot 2^4 \cdot 17\ ∧\ b = 18 \NR
\NC a = 18 \cdot 2^4\ ∧\ b = 18 \cdot 17 \NR
\NC a = 18 \cdot 17\ ∧\ b = 18 \cdot 2^4 \NR
\NC a = 18\ ∧\ b = 18 \cdot 17 \cdot 2^4 \NR
\stopalign \stopformula
\page


\centerline {Задание №7}
\startformula \startalign[align=left]
\NC a = 1716,\ b = -72,\ c = 124 \quad НОД(a,b,c),\ НОК(b,c) =\ ? \NR
\NC a = 1716 = 2^2 \cdot 3 \cdot 11 \cdot 13 \NR
\NC b = -72 = -1 \cdot 2^3 \cdot 3^2 \NR
\NC c = 124 = 2^2 \cdot 31 \NR
\NC НОД(a,b,c) = 2^2 = 4 \NR
\NC НОК(b,c) = 2^3 \cdot 3^2 \cdot 31 = 2232 \NR
\stopalign \stopformula

\stoptext
