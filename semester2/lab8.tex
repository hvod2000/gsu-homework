\definecolor[headingcolor][r=1, b=0.4]
\definecolor[solutionheadercolor][r=0.9, g=0.5]
\setuphead[subject][color=headingcolor, indentnext=yes]
\setuphead[subsubject][color=solutionheadercolor, indentnext=yes]
\setuphead[subject][style={\ss\bfa}, before={\bigskip\bigskip\bigskip\bigskip}, after={}]
\setuphead[subsubject][style={\ss\bfa}, before={}, after={}]

\setuppapersize[A4]
\setuplayout[backspace=2cm, topspace=1cm, header=1cm, bottomspace=1cm,
    footer=1cm, width=middle, height=middle]
\setupbodyfont[libertinus, 14pt]
\mainlanguage[ru]
\setupwhitespace[medium]
\setupindenting[medium, yes]

\definemathmatrix[lmatrix][simplecommand=lmat, left={\left\{\,}, right={\,\right. }]
\definemathmatrix[umatrix][simplecommand=umat, left={\left[\,}, right={\,\right. }]
\definemathmatrix[pmatrix][simplecommand=pmat, left={\left(\,}, right={\,\right) }]
\definemathmatrix[dmatrix][simplecommand=dmat, left={\left|\,}, right={\,\right| }]
\definemathmatrix[gmatrix][simplecommand=grid, left={\left.\,}, right={\,\right. }]
\definemathcommand[arctg][nolop]{\mfunction{arctg}}

\setupindenting[medium,yes]
%\setupformulas[align=right]
\starttext
\setuppagenumbering[state=stop]
\centerline { Гомельский государственный университет имени Франциска Скорины }
\centerline { факультет математики и технологий программирования }
\vfill \vfill
\centerline { Лабораторная работа №8 }
\centerline { «Квадратичные формы» }
\vfill \vfill
{\leftskip 0.55\hsize \noindent
  Выполнил:\\Хамков Владислав (ПИ-11)\\
  Проверил:\\Васильев Александр Федорович}
\vfill
\centerline { Июнь 2022 }
\page
\setuppagenumber[number=1]
\setupfootertexts[\qquad \date \hfill Страница \pagenumber\ / \lastpagenumber \qquad]

\subject {Задание №1}
Запишите матрицы квадратичных форм $F_1$, $F_2$. Вычислите ранги этих квадратичных форм.
\startformula F_1(x, y, z) = x^2 - 2xy + 2z^2 + 4yz + 5z^2 \stopformula
\startformula F_2(x_1, x_2, x_3, x_4) = x_1 x_2 + x_1 x_3 + x_1 x_4 + x_2 x_3 + x_2 x_4 \stopformula
\subsubject {Решение}
\startformula
M_{F_1} = \pmat{1, -1, 0; -1, 2, 2; 0, 2, 5} \qquad \qquad
M_{F_2} = \pmat{
	0, \frac12, \frac12, \frac12;
	\frac12, 0, \frac12, \frac12;
	\frac12, \frac12, 0, 0;
	\frac12, \frac12, 0, 0 }
\stopformula
\startformula \text{rang}\ F_1 = \text{rang}\ M_{F_1} = \text{rang}\ \pmat{1, -1, 0; -1, 2, 2; 0, 2, 5} = 3 \stopformula
\startformula
\text{rang}\ F_2 = \text{rang}\ M_{F_2} = \text{rang}\
\pmat{ 0, \frac12, \frac12, \frac12; \frac12, 0, \frac12, \frac12; \frac12, \frac12, 0, 0; \frac12, \frac12, 0, 0 } = 3
\stopformula


\subject {Задание №2}
Методом Лагранжа приведите квадратичные формы $G_1$ и $G_2$ к каноническому виду.
Укажите невырожденное линейное преобразование переменных, приводящее к этому виду. Сделайте проверку.
\startformula G_1 = x_1^2 − 2x_2^2 + x_3^2 + 2x_1 x_2 + 4x_1 x_3 + 2x_2 x_3 \stopformula
\startformula G_2 = 2x_1 x_3 − 4x_2 x_3 \stopformula
\subsubject {Решение}
TODO


\subject {Задание №3}
Определите индекс инерции квадратичных форм $F_1$, $G_1$, $F_2$, $G_2$.
\startformula F_1(x, y, z) = x^2 - 2xy + 2z^2 + 4yz + 5z^2 \stopformula
\startformula G_1 = x_1^2 − 2x_2^2 + x_3^2 + 2x_1 x_2 + 4x_1 x_3 + 2x_2 x_3 \stopformula
\startformula F_2(x_1, x_2, x_3, x_4) = x_1 x_2 + x_1 x_3 + x_1 x_4 + x_2 x_3 + x_2 x_4 \stopformula
\startformula G_2 = 2x_1 x_3 − 4x_2 x_3 \stopformula
\subsubject {Решение}
TODO


\subject {Задание №4}
Найдите ортогональную матрицу $T$ такую, что $T AT^{−1}$ --- диагональная матрица,
где $A$ --- матрица квадратичной формы $F$. Сделайте проверку.
\startformula F = 17x_1^2 + 17x_2^2 + 11x_3^2 − 16x_1 x_2 + 8x_1 x_3 − 8x_2 x_3 \stopformula
\subsubject {Решение}
TODO


\subject {Задание №5}
Найдите ортогональное преобразование переменных,
приводящее квадратичную форму $F(x_1, x_2, x_3)$ к каноническому виду.
\startformula F = 8x_1^2 − 7x_2^2 + 8x_3^2 + 8x_1 x_2 − 2x_1 x_3 + 8x_2 x_3 \stopformula
\subsubject {Решение}
TODO


\subject {Задание №6}
Найдите нормальный вид над полем ${\Bbb R}$ и над полем ${\Bbb C}$,
а также невырожденное линейное преобразование переменных,
приводящее к этому виду, для квадратичных форм $F_1$ и $F_2$.
\startformula F_1 = 17x_1^2 + 17x_2^2 + 11x_3^2 − 16x_1 x_2 + 8x_1 x_3 − 8x_2 x_3 \stopformula
\startformula F_2 = 8x_1^2 − 7x_2^2 + 8x_3^2 + 8x_1 x_2 − 2x_1 x_3 + 8x_2 x_3 \stopformula
\subsubject {Решение}
TODO


\subject {Задание №7}
Вычислите положительный и отрицательный индексы инерции квадратичных форм $G_1$, $G_2$.
Являются ли эти формы положительно определенными (отрицательно определенными)?
\startformula G_1 = x_1^2 − 2x_2^2 + x_3^2 + 2x_1 x_2 + 4x_1 x_3 + 2x_2 x_3 \stopformula
\startformula G_2 = 2x_1 x_3 − 4x_2 x_3 \stopformula
\subsubject {Решение}
TODO


\subject {Задание №8}
При каких значениях $λ$ квадратичная форма $F$ является положительно (отрицательно) определенной?
\startformula F = 2x_1^2 + x_2^2 + 3x_3^2 + 2λx_1 x_2 + 2x_1 x_3 \stopformula
\subsubject {Решение}
TODO


\stoptext
