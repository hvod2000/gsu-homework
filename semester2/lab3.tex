\definecolor[headingcolor][r=1, b=0.5]
\definecolor[solutionheadercolor][r=0.9, b=1]
\setuphead[subject][color=headingcolor, indentnext=yes]
\setuphead[subsubject][color=solutionheadercolor, indentnext=yes]
\setuphead[subject][style={\ss\bfa}, before={\bigskip\bigskip\bigskip\bigskip}, after={}]
\setuphead[subsubject][style={\ss\bfa}, before={}, after={}]

\setuppapersize[A4]
\setuplayout[backspace=2cm, topspace=1cm, header=1cm, bottomspace=1cm,
    footer=1cm, width=middle, height=middle]
\setupbodyfont[libertinus, 14pt]
\mainlanguage[ru]
\setupwhitespace[medium]
\setupindenting[medium, yes]

\definemathmatrix[lmatrix][simplecommand=lmat, left={\left\{\,}, right={\,\right. }]
\definemathmatrix[umatrix][simplecommand=umat, left={\left[\,}, right={\,\right. }]
\definemathmatrix[pmatrix][simplecommand=pmat, left={\left(\,}, right={\,\right) }]
\definemathmatrix[dmatrix][simplecommand=dmat, left={\left|\,}, right={\,\right| }]
\definemathmatrix[gmatrix][simplecommand=grid, left={\left.\,}, right={\,\right. }]
\definemathcommand[arctg][nolop]{\mfunction{arctg}}

\setupindenting[medium,yes]
%\setupformulas[align=right]
\starttext
\setuppagenumbering[state=stop]
\centerline { Гомельский государственный университет имени Франциска Скорины }
\centerline { факультет математики и технологий программирования }
\vfill \vfill
\centerline { Лабораторная работа №3 }
\centerline { «Подпространство линейного пространства» }
\vfill \vfill
{\leftskip 0.55\hsize \noindent
  Выполнил:\\Хамков Владислав (ПИ-11)\\
  Проверил:\\Васильев Александр Федорович}
\vfill
\centerline { Май 2022 }
\page

\setuppagenumbering[state=start]
\setuppagenumber[number=1]
\setuppagenumbering[location={footer,center}]
\setupfootertexts[\qquad \date \hfill Страница \pagenumber\ / \lastpagenumber \qquad]

\subject {Задание №1}
Являются ли подпространствами линейного пространства $M(n, {\Bbb C})$ следующее множество:
множество всех вырожденных матриц.
\subsubject {Решение}
Нет, множество всех вырожденных матриц не является линейным пространством, т.к 
сумма двух вырожденных матриц не всегда является вырожденной матрицей. Например,
\startformula
\pmat{1, 0; 0, 0} + \pmat{0, 0; 0, 1} = \pmat{1, 0; 0, 1},\ \dmat{1, 0; 0, 1} = 1
\stopformula


\subject {Задание №2}
Является ли подпространством линейного пространства $C_{[a,b]}$ множество всех функций $f(x)$, удовлетворяющих
следующим условиям?
\startformula
	f(x) = a + bx, \ a,b ∈ {\Bbb R}
\stopformula
\subsubject {Решение}
Покажем, что сумма любых двух элементов этого множества так же является элементом этого множества:
\startformula
	f_1(x) + f_2(x) = (a_1 + b_1x) + (a_2 + b_2x) = (a_1 +a_2) + (b_1+b_2)x = a_3 + b_3x = f_3(x)
\stopformula
Покажем, что произведение элемента этого множества на число $γ \in {\Bbb R}$ так же является элементом этого множества:
\startformula
	γ f_1(x) = γ \cdot (a_1 + b_1x) =  γa_1 + (γb_1)x = a_2 + b_2x = f_2(x)
\stopformula
Таким образом, вышеуказанное множество является замкнутым относительно операций сложения и умножения на число.
Следовательно, оно является подпространством линейного пространства $C_{[a,b]}$.


\subject {Задание №3}
Найдите размерность и базис линейной оболочки $L(a_1, a_2, a_3, a_4)$ системы векторов из ${\Bbb R}^4$.
\startformula
	a_1=(1,1,1,1)\quad a_2=(1,1,-1,-1)\quad a_3=(2,2,0,0)\quad a_4=(1,-1,-1,0)
\stopformula
\subsubject {Решение}
Для того, чтобы найти базис линейной оболочки $L(a_1, a_2, a_3, a_4)$ будем из системы векторов
$a_1$, $a_2$, $a_3$, $a_4$ последовательно удалять те векторы, которые линейно выражаются через предыдущие.
Нетрудно заметить, что $a_1$ и $a_2$ являются линейно независимыми векторами.
Проверим на независимость $a_1$, $a_2$, $a_3$:
\startformula \startalign
\NC α_1 a_1 + α_2 a_2 + α_3 a_3 \NC = 0 \NR
\NC α_1 (1,1,1,1) + α_2 (1,1,-1,-1) + α_3 (2,2,0,0) \NC = 0 \NR
\stopalign \stopformula
\startformula \startcases
α_1 + α_2 + 2α_3 = 0 \NR
α_1 + α_2 + 2α_3 = 0 \NR
α_1 - α_2 = 0 \NR
α_1 - α_2 = 0 \NR
\stopcases 
\Rightarrow
\startcases
α_1 = -α_3 \NR
α_2 = -α_3 \NR
\stopcases 
\stopformula
У системы есть ненулевые решения, поэтому вектор $a_3$ линейно выражается из $a_1$ и $a_2$.
Проверим на независимость $a_1$, $a_2$, $a_4$:
\startformula \startalign
\NC α_1 a_1 + α_2 a_2 + α_3 a_4 \NC = 0 \NR
\NC α_1 (1,1,1,1) + α_2 (1,1,-1,-1) + α_3 (1,-1,-1,0) \NC = 0 \NR
\stopalign \stopformula
\startformula \startcases
α_1 + α_2 + α_3 = 0 \NR
α_1 + α_2 - α_3 = 0 \NR
α_1 - α_2 - α_3 = 0 \NR
α_1 - α_2 = 0 \NR
\stopcases 
\Rightarrow
\startcases
α_1 = 0 \NR
α_2 = 0 \NR
α_3 = 0 \NR
α_4 = 0 \NR
\stopcases 
\stopformula
У системы нет ненулевых решений. Следовательно, вектора $a_1$, $a_2$, $a_4$ являются линейно независимыми и
образуют базис линейной оболочки $L(a_1, a_2, a_3, a_4)$. А её размерность равна трём.


\subject {Задание №4}
Найдите размерность и базис линейной оболочки $L(a_1, a_2, a_3)$ векторов из ${\Bbb R}_2[x]$.
\startformula
	a_1=3x^2-1 \qquad a_2=2x+1 \qquad a_3=x^2+x-1
\stopformula
\subsubject {Решение}
Для того, чтобы найти базис линейной оболочки $L(a_1, a_2, a_3)$ будем из системы векторов
$a_1$, $a_2$, $a_3$ последовательно удалять те векторы, которые линейно выражаются через предыдущие.
Нетрудно заметить, что $a_1$ и $a_2$ являются линейно независимыми векторами.
Проверим на независимость $a_1$, $a_2$, $a_3$:
\startformula \startalign
\NC α_1 a_1 + α_2 a_2 + α_3 a_3 \NC = 0 \NR
\NC α_1 (3x^2-1) + α_2 (2x+1) + α_3 (x^2+x-1) \NC = 0 \NR
\stopalign \stopformula
\startformula \startcases
3α_1 + α_3 = 0 \NR
2α_2 + α_3 = 0 \NR
-α_1 + α_2 - α_3 = 0 \NR
\stopcases 
\Rightarrow
\startcases
α_1 = 0 \NR
α_2 = 0 \NR
α_3 = 0 \NR
\stopcases 
\stopformula
У системы нет ненулевых решений. Следовательно, вектора $a_1$, $a_2$, $a_3$ являются линейно независимыми и
образуют базис линейной оболочки $L(a_1, a_2, a_3)$. А её размерность равна трём.


\subject {Задание №5}
Найдите размерность и базис суммы, а также размерность пересечения подпространств $A=L(a_1, a_2, a_3)$ и
$B=L(b1, b2)$ линейного пространства ${\Bbb R}^3$.
Принадлежит ли вектор $x$ пространствам $L(a_1, a_2, a_3)$, $L(b_1, b_2)$?
\startformula
	a_1=(0,1,3) \qquad a_2=(-1,2,0) \qquad a_3=(-1,3,3)
\stopformula \startformula
	b_1=(-1,2,3) \qquad b_2=(3,2,1) \qquad x=(0,-1,2)
\stopformula
\subsubject {Решение}
Проверим на независимость $a_1$, $a_2$, $a_3$:
\startformula \startalign
\NC α_1 a_1 + α_2 a_2 + α_3 a_3 \NC = 0 \NR
\NC α_1 (0,1,3) + α_2 (-1,2,0) + α_3 (-1,3,3) \NC = 0 \NR
\stopalign \stopformula
\startformula \startcases
0   -α_2  -α_3 = 0 \NR
α_1+ 2α_2+ 3α_3 =0\NR
3α_1 + 0 + 3α_3 =0\NR
\stopcases 
\Rightarrow
\startcases
α_1 = -α_3 \NR α_2 = -α_3 \NR
\stopcases 
\stopformula
У системы есть ненулевые решения. Следовательно, вектор $a_3$ линейно выражается из $a_1$ и $a_2$.
Проверим на независимость $a_1$, $a_2$, $b_1$:
\startformula \startalign
\NC α_1 a_1 + α_2 a_2 + α_3 b_1 \NC = 0 \NR
\NC α_1 (0,1,3) + α_2 (-1,2,0) + α_3 (-1,2,3) \NC = 0 \NR
\stopalign \stopformula
\startformula \startcases
0   -α_2  -α_3 = 0 \NR
α_1+ 2α_2+ 2α_3 =0\NR
3α_1 + 0 + 3α_3 =0\NR
\stopcases 
\Rightarrow
\startcases
α_1 = 0 \NR α_2 = 0 \NR α_3 = 0 \NR
\stopcases 
\stopformula
У системы нет ненулевых решений. Следовательно, вектора $a_1$, $a_2$, $b_1$ являются линейно независимыми и
образуют базис линейной оболочки $L(a_1, a_2, a_3, b_1, b_2)$. А её размерность равна трём.
Проверять на независимость $a_1$, $a_2$, $b_1$, $b_2$ нет смысла, т.к. четыре вектора не могут быть линейно независимыми
в трёхмерном пространстве. Вычислим размерности суммы и разности $A$ и $B$:
\startformula \startalign
\NC \text{dim}(A + B) \NC = \text{dim}(L(a_1, a_2, a_3, b_1, b_2)) = 3 \NR
\NC \text{dim}(A ∩ B) \NC = \text{dim}(A) + \text{dim}(B) - \text{dim}(A + B) = 2 + 2 - 3 = 1 \NR
\stopalign \stopformula
Проверим, принадлежит ли вектор $x$ пространству $L(a_1, a_2, a_3)$:
\startformula \startalign
\NC α_1 a_1 + α_2 a_2 \NC = x \NR
\NC α_1 (0,1,3) + α_2 (-1,2,0) \NC = (0,-1,2) \NR
\stopalign \stopformula
\startformula \startcases
0   -α_2 = 0 \NR
α_1+ 2α_2 =-1\NR
3α_1 + 0  =-2\NR
\stopcases 
\stopformula
У полученной системы нет решений, следовательно, вектор $x$ не принадлежит пространству $L(a_1, a_2, a_3)$:
Проверим, принадлежит ли вектор $x$ пространству $L(b_1, b_2)$:
\startformula \startalign
\NC α_1 b_1 + α_2 b_2 \NC = x \NR
\NC α_1 (-1,2,3) + α_2 (3,2,1) \NC = (0,-1,2) \NR
\stopalign \stopformula
\startformula \startcases
-α_1 + 2α_2 = 0 \NR
2α_1 + 2α_2 = -1 \NR
3α_1 + α_2 = 2   \NR
\stopcases 
\stopformula
У полученной системы нет решений, следовательно, вектор $x$ не принадлежит пространству $L(b_1, b_2)$.


\subject {Задание №6}
Пусть в пространстве ${\Bbb R}^4$ заданы подпространства $A = L(a_1, a_2)$ и $B = L(a_3, a_4)$.
Докажите, что ${\Bbb R}^4 = A \oplus B$.
Представьте вектор $x$ в виде $x = x_1 + x_2$, где $x_1 ∈ A$, $x_2 ∈ B$.
\startformula
	a_1=(1,2,3,-1) \quad a_2=(2,3,-4,4) \quad a_3=(3,-4,2,1)
\stopformula \startformula
	a_4 = (2, −3, 1, −4) \qquad x = (0, 9, −3, 2)
\stopformula
\subsubject {Решение}
Проверим на линейную независимость $a_1$, $a_2$, $a_3$, $a_4$:
\startformula
α_1 a_1 + α_2 a_2 + α_3 a_3 + α_4 a_4 = 0
\stopformula
\startformula \startcases
 α_1 + 2α_2 + 3α_3 + 2α_4 = 0 \NR
2α_1 + 3α_2  -4α_3  -3α_4 = 0 \NR
3α_1  -4α_2 + 2α_3 +  α_4 = 0 \NR
-α_1 + 4α_2 +  α_3  -4α_4 = 0 \NR
\stopcases 
\Rightarrow
\startcases
α_1 = 0 \NR α_2 = 0 \NR α_3 = 0 \NR α_4 = 0 \NR
\stopcases 
\stopformula
У системы нет ненулевых решений. Следовательно, вектора $a_1$, $a_2$, $a_3$, $a_4$ являются линейно независимыми и
образуют базис линейной оболочки $L(a_1, a_2, a_3, a_4) = A \oplus B $. А её размерность равна четырём.
Поскольку $\text{dim}({\Bbb R}^4) = 4$ и $\text{dim}(A \oplus B) = 4$, то ${\Bbb R}^4 = A \oplus B$,
что и требовалось доказать.
\par
Представим вектор $x$ в виде $x = x_1 + x_2$, где $x_1 ∈ A$, $x_2 ∈ B$:
\startformula
α_1 a_1 + α_2 a_2 + α_3 a_3 + α_4 a_4 = x
\stopformula
\startformula \startcases
 α_1 + 2α_2 + 3α_3 + 2α_4 = 0 \NR
2α_1 + 3α_2  -4α_3  -3α_4 = 9 \NR
3α_1  -4α_2 + 2α_3 +  α_4 = -3 \NR
-α_1 + 4α_2 +  α_3  -4α_4 = 2 \NR
\stopcases 
\Rightarrow
\startcases
α_1 = 1 \NR α_2 = 1 \NR α_3 = -1 \NR α_4 = 0 \NR
\stopcases 
\Rightarrow
\startcases
x = x_1 + x_2 \NR
x_1 = a_1 + a_2 \in A\NR
x_2 = -a_3 \in B \NR
\stopcases 
\stopformula




\subject {Задание №7}
Линейная оболочка системы векторов $a_1$, $a_2$, $a_3$ является пространством решений
некоторой однородной системы линейных уравнений. Найдите эту систему уравнений.
\startformula
	a_1=(1,1,1,1)\qquad a_2=(1,1,-1,-1)\qquad a_3=(2,2,0,0)
\stopformula
\subsubject {Решение}
Проверим на линейную независимость вектора $a_1$, $a_2$, $a_3$:
\startformula \startalign
\NC α_1 a_1 + α_2 a_2 + α_3 a_3 \NC = 0 \NR
\NC α_1 (1,1,1,1) + α_2 (1,1,-1,-1) + α_3 (2,2,0,0) \NC = 0 \NR
\stopalign \stopformula
\startformula \startcases
α_1 + α_2 + 2α_3 = 0 \NR
α_1 + α_2 + 2α_3 = 0 \NR
α_1 - α_2 = 0 \NR
α_1 - α_2 = 0 \NR
\stopcases 
\Rightarrow
\startcases
α_1 = -α_3 \NR
α_2 = -α_3 \NR
\stopcases 
\stopformula
У системы есть ненулевые решения, поэтому вектор $a_3$ линейно выражается из $a_1$ и $a_2$.
Вектора $a_1$ и $a_2$ линейно независимы, следовательно, они являются базисом пространства решений.
Пространство решений можно записать таким образом:
\startformula
	(x_1, x_2, x_3, x_4) = α_1 a_1 + α_2 a_2 = (α_1 + α_2,\ α_1 + α_2,\ α_1 - α_2,\ α_1 - α_2) = (a, a, b, b)
\stopformula
Вышеуказанное уравнение эквивалентно следующей однородной системе:
\startformula \startcases
x_1 = x_2 \NR
x_3 = x_4 \NR
\stopcases \text{или} \quad \startcases
x_1 - x_2 = 0 \NR
x_3 - x_4 = 0 \NR
\stopcases \stopformula
что и требовалось найти.

\stoptext
