\definecolor[headingcolor][r=1, b=0.4]
\definecolor[solutionheadercolor][r=0.9, b=0.5]
\setuphead[subject][color=headingcolor, indentnext=yes]
\setuphead[subsubject][color=solutionheadercolor, indentnext=yes]
\setuphead[subject][style={\ss\bfa}, before={\bigskip\bigskip\bigskip\bigskip}, after={}]
\setuphead[subsubject][style={\ss\bfa}, before={}, after={}]

\setuppapersize[A4]
\setuplayout[backspace=2cm, topspace=1cm, header=1cm, bottomspace=1cm,
    footer=1cm, width=middle, height=middle]
\setupbodyfont[libertinus, 14pt]
\mainlanguage[ru]
\setupwhitespace[medium]
\setupindenting[medium, yes]

\definemathmatrix[lmatrix][simplecommand=lmat, left={\left\{}, right={\right. }]
\definemathmatrix[umatrix][simplecommand=umat, left={\left[}, right={\right. }]
\definemathmatrix[pmatrix][simplecommand=pmat, left={\left(}, right={\right) }]
\definemathmatrix[dmatrix][simplecommand=dmat, left={\left|}, right={\right| }]
\definemathmatrix[gmatrix][simplecommand=grid, left={\left.}, right={\right. }]
\definemathcommand[arctg][nolop]{\mfunction{arctg}}

\setupindenting[medium,yes]
%\setupformulas[align=right]
\starttext
\setuppagenumbering[state=stop]
\centerline { Гомельский государственный университет имени Франциска Скорины }
\centerline { факультет математики и технологий программирования }
\vfill \vfill
\centerline { Лабораторная работа №5 }
\centerline { «Строение линейного оператора» }
\vfill \vfill
{\leftskip 0.55\hsize \noindent
  Выполнил:\\Хамков Владислав (ПИ-11)\\
  Проверил:\\Васильев Александр Федорович}
\vfill
\centerline { Май 2022 }
\page
\setuppagenumber[number=1]
\setupfootertexts[\qquad \date \hfill Страница \pagenumber\ / \lastpagenumber \qquad]

\subject {Задание №1}
Найдите собственные значения и собственные векторы линейных операторов,
заданных в некотором базисе линейного пространства над полем ${\Bbb R}$
и над полем ${\Bbb C}$ следующими матрицами.
\startformula
A = \pmat{3, -2; 4, -1} \qquad\qquad B = \pmat{1, -2, -1; -1, 1, 1; 1, 0, -1}
\stopformula
\subsubject {Решение}
Для нахождения собственных значений матрицы $A$ решим относительно $λ$ уравнение $|A - λE| = 0$
\startformula \startalign
\NC \dmat{3 - λ, -2; 4, -1-λ} \NC = 0 \NR
\NC λ^2 - 2λ + 5 \NC = 0 \NR
\stopalign \stopformula
\startformula
λ_1 = -2i + 1 \qquad\qquad λ_2 = 2i + 1
\stopformula
К сожалению, собственные значения матрицы $A$ не принадлежат полю ${\Bbb R}$, т.е.
у линейного оператора, заданного матрицей $A$, нет ни собственных значений, ни собственных векторов.
\par Найдём собственные значений матрицы $B$
\startformula \startalign
\NC \dmat{1 - λ, -2, -1; -1, 1 - λ, 1; 1, 0, -1 - λ} \NC = 0 \NR
\NC -λ^3 + λ^2 + 2*λ \NC = 0 \NR
\stopalign \stopformula
\startformula
λ_1 = 0 \qquad λ_2 = -1 \qquad λ_3 = 2
\stopformula
Найдём собственный вектор матрицы $B$ соответствующий собственному значению $λ=0$
\startformula \startalign
\NC \pmat{ x_1, x_2, x_3 } \pmat{1 - λ, -2, -1; -1, 1 - λ, 1; 1, 0, -1 - λ} \NC = 0 \NR
\NC \pmat{ x_1, x_2, x_3 } \NC = \pmat{ c, 2c, c} \NR
\stopalign \stopformula
Найдём собственный вектор матрицы $B$ соответствующий собственному значению $λ=-1$
\startformula \startalign
\NC \pmat{ x_1, x_2, x_3 } \pmat{1 + 1, -2, -1; -1, 1 + 1, 1; 1, 0, -1 + 1} \NC = 0 \NR
\NC \pmat{ x_1, x_2, x_3 } \NC = \pmat{ c, c, -c} \NR
\stopalign \stopformula
Найдём собственный вектор матрицы $B$ соответствующий собственному значению $λ=2$
\startformula \startalign
\NC \pmat{ x_1, x_2, x_3 } \pmat{1 - 2, -2, -1; -1, 1 - 2, 1; 1, 0, -1 - 2} \NC = 0 \NR
\NC \pmat{ x_1, x_2, x_3 } \NC = \pmat{ c, -2c, -c} \NR
\stopalign \stopformula
Таким образом, собственными векторами матрицы $B$ являются вектора вида:
\startformula
\pmat{ c, 2c, c } \quad \pmat{ c, c, -c } \quad \pmat{ c, -2c, -c}
\stopformula


\subject {Задание №2}
Пусть $f$ --- линейный оператор пространства $V$ над полем $P$. Докажите следующее утверждение.
Линейная оболочка любой системы собственных векторов оператора $f$ инвариантна относительно $f$.
\subsubject {Решение}
Заметим, что по определению собственного вектора $f$, его линейная оболочка инвариантна относительно $f$.
При этом, линейная оболочка системы собственных векторов является суммой линейных оболочек этих векторов.
Согласно лемме 5.7 стр.73, сумма инвариантных относительно $f$ подпространств
является инвариантной относительно $f$ подпространством $V$.
Следовательно, линейная оболочка любой системы собственных векторов оператора $f$ инвариантна относительно $f$,
что и требовалось доказать.


\subject {Задание №3}
Какие из матриц линейных операторов в пространстве $V$ над ${\Bbb R}$ можно
привести к диагональному виду путем перехода к новому базису?
Найдите этот базис и соответствующую диагональную матрицу.
\startformula
A = \pmat{1, -1; -4, 4} \qquad
B = \pmat{5, -1, -4; -12, 5, 12; 10, -3, -9} \qquad
C = \pmat{-1, 3, -1; -3, 5, -1; -3, 3, 1}
\stopformula
\subsubject {Решение}
Известно, что оператор $f$ $n$-мерного пространства $V$ над полем $P$ диагонализируем тогда и только тогда,
когда все корни характеристического многочлена оператора $f$ принадлежат полю $P$ и различны.

Найдём собственные значения и вектора матрицы $A$
\startformula \dmat{1 - λ, -1; -4, 4-λ} = 0 \quad \Rightarrow \quad λ^2 - 5*λ = 0 \stopformula
\startformula λ_1 = 0 \qquad\qquad λ_2 = 5 \stopformula
\startformula λ = 0 \Rightarrow \pmat{ x_1, x_2 } \pmat{1 - 0, -1; -4, 4 - 9} = 0
\Rightarrow \pmat{ x_1, x_2 } = \pmat{ 4c, c} \stopformula
\startformula λ = 5 \Rightarrow \pmat{ x_1, x_2 } \pmat{1 - 5, -1; -4, 4 - 5} = 0
\Rightarrow \pmat{ x_1, x_2 } = \pmat{ c, -c} \stopformula
Таким образом мы выяснили, что линейный оператор, соответствующий матрице $A$, диагонализируем при
переходе к базису
\startformula e_1 = \pmat{4, 1} \qquad\qquad e_2 = \pmat{1, -1} \stopformula
А сама диагональная матрица данного оператора имеет вид
\startformula \pmat{ 0, 0; 0, 5 } \stopformula

Найдём собственные значения и вектора матрицы $B$
\startformula \dmat{5-λ, -1, -4; -12, 5-λ, 12; 10, -3, -9-λ} = 0 \quad \Rightarrow \quad -λ^3 + λ^2 + λ - 1 = 0 \stopformula
\startformula λ_1 = -1 \qquad λ_2 = 1 \qquad λ_3 = -1 \stopformula
Корни характеристического многочлена оператора $f_B$ совпали, следовательно,
линейный оператор, соответствующий матрице $A$, не диагонализируем.

Найдём собственные значения и вектора матрицы $C$
\startformula \dmat{-1-λ, 3, -1; -3, 5-λ, -1; -3, 3, 1-λ} = 0 \quad \Rightarrow \quad -λ^3 + 5*λ^2 - 8*λ + 4 = 0 \stopformula
\startformula λ_1 = 1 \qquad λ_2 = 2 \qquad λ_3 = 2 \stopformula
Корни характеристического многочлена оператора $f_C$ совпали, следовательно,
линейный оператор, соответствующий матрице $C$, не диагонализируем.


\subject {Задание №4}
Найдите жорданову нормальную форму матриц.
\startformula
A = \pmat{1, 0; 1, 0} \qquad
B = \pmat{1, -3, 4; 4, -7, 8; 6, -7, 7} \qquad
C = \pmat{4, 1, 1, 1; -1, 2, -1, -1; 6, 1, -1, 1; -6, -1, 4, 2}
\stopformula
\subsubject {Решение}
Для нахождения жорданову нормальной формы нам потребуется найти корни характеристического многочлена.
Составим характеристический многочлен матрицы $A$ и найдем его корни.
\startformula \dmat{1 - λ, 0; 1, 0 - λ} = 0 \Rightarrow
λ^2 - λ = 0 \stopformula
\startformula λ_1 = 1 \qquad\qquad λ_2 = 0 \stopformula
Характеристический многочлен имеет корни кратности 1, поэтому жорданова нормальная форма состоит
только из клеток единичного размера и имеет вид:
\startformula \pmat{1, 0; 0, 0} \stopformula

Составим характеристический многочлен матрицы $B$ и найдем его корни.
\startformula \dmat{1 - λ, -3, 4; 4- λ, -7, 8; 6, -7, 7 - λ} = 0 \Rightarrow
λ^3 - λ^2 - 5*λ - 3 = 0 \stopformula
\startformula λ_1 = 3 \qquad λ_2 = -1 \qquad λ_3 = -1 \stopformula
Характеристический многочлен имеет корень $λ=-1$ кратности 2.
Определим число клеток Жордана, соответствующих этому собственному значению.
Для этого вычислим ранг матрицы $B − λE$.
\startformula 
\text{rang}\ \left(\pmat{1, -3, 4; 4, -7, 8; 6, -7, 7} - \pmat{-1,0,0;0,-1,0;0,0,-1} \right) = 2
\stopformula
$n - \text{rang}(B - λE) = 3 - 2 = 1$. Таким образом, жорданова нормальная форма содержит только одну клетку,
соответствующую собственному значению $-1$, и имеет вид:
\startformula \pmat{3,0,0; 0,-1,1; 0,0,-1} \stopformula


Составим характеристический многочлен матрицы $C$ и найдем его корни.
\startformula \dmat{4 - λ, 1, 1, 1; -1, 2-λ, -1, -1; 6, 1, -1-λ, 1; -6, -1, 4, 2-λ} = 0 \Rightarrow
λ^4 - 7λ^3 + 9λ^2 + 27λ - 54 = 0 \stopformula
\startformula λ_1 = -2 \quad λ_2 = 3 \quad λ_3 = 3 \quad λ_4 = 3 \stopformula
Характеристический многочлен имеет корень $λ=3$ кратности 3.
Определим число клеток Жордана, соответствующих этому собственному значению.
Для этого вычислим ранг матрицы $C − λE$.
\startformula 
\text{rang}\ \left(\pmat{4, 1, 1, 1; -1, 2, -1, -1; 6, 1, -1, 1; -6, -1, 4, 2} - 3 E_4 \right) = 2
\stopformula
$n - \text{rang}(C - λE) = 4 - 2 = 2$. Таким образом, жорданова нормальная форма содержит две клетки,
соответствующие собственному значению $3$, размеры которых в сумме равны трём.
Т.е. их размеры равны $1$ и $2$, а сама жорданова нормальная форма матрицы $C$ имеет вид:
\startformula \pmat{-2,0,0,0; 0,3,1,0; 0,0,3,0; 0,0,0,3} \stopformula


\stoptext
