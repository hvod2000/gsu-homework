\definecolor[headingcolor][r=1]
\definecolor[solutionheadercolor][r=0.5, b=1]
\setuphead[subject][color=headingcolor, indentnext=yes]
\setuphead[subsubject][color=solutionheadercolor, indentnext=yes]
\setuphead[subject][style={\ss\bfa}, before={\bigskip\bigskip\bigskip\bigskip}, after={}]
\setuphead[subsubject][style={\ss\bfa}, before={}, after={}]

\setuppapersize[A4]
\setuplayout[backspace=2cm, topspace=1cm, header=1cm, bottomspace=1cm,
    footer=1cm, width=middle, height=middle]
\setupbodyfont[libertinus, 14pt]
\mainlanguage[ru]
\setupwhitespace[medium]
\setupindenting[medium, yes]

\definemathmatrix[lmatrix][simplecommand=lmat, left={\left\{\,}, right={\,\right. }]
\definemathmatrix[umatrix][simplecommand=umat, left={\left[\,}, right={\,\right. }]
\definemathmatrix[pmatrix][simplecommand=pmat, left={\left(\,}, right={\,\right) }]
\definemathmatrix[dmatrix][simplecommand=dmat, left={\left|\,}, right={\,\right| }]
\definemathmatrix[gmatrix][simplecommand=grid, left={\left.\,}, right={\,\right. }]
\definemathcommand[arctg][nolop]{\mfunction{arctg}}

\setupindenting[medium,yes]
%\setupformulas[align=right]
\starttext
\setuppagenumbering[state=stop]
\centerline { Гомельский государственный университет имени Франциска Скорины }
\centerline { факультет математики и технологий программирования }
\vfill \vfill
\centerline { Лабораторная работа №4 }
\centerline { «Линейные отображения линейных пространств» }
\vfill \vfill
{\leftskip 0.55\hsize \noindent
  Выполнил:\\Хамков Владислав (ПИ-11)\\
  Проверил:\\Васильев Александр Федорович}
\vfill
\centerline { Май 2022 }
\page

\setuppagenumbering[state=start]
\setuppagenumber[number=1]
\setuppagenumbering[location={footer,center}]
\setupfootertexts[\qquad \date \hfill Страница \pagenumber\ / \lastpagenumber \qquad]

\subject {Задание №1}
В трехмерном линейном пространстве с базисом $e_1$, $e_2$, $e_3$ отображение $φ$ переводит произвольный вектор
\startformula c = c_1 e_1 + c_2 e_2 + c_3 e_3 \stopformula
в вектор 
\startformula 
	φ(c) = (ac_1 − bc_2)e_1 + (c_1 − ac_2)e_2 + (c_2 − bc_3)e_3,\ \text{где}\ 
	a = 2,\ b = 1
\stopformula
Является ли это отображение линейным оператором?
В случае положительного ответа: запишите матрицу линейного оператора $φ$ в базисе $e_1$, $e_2$, $e_3$,
найдите ядро линейного оператора $φ$, найдите ранг и дефект оператора $φ$.
\subsubject {Решение}
Проверим условие аддитивности:
\startformula \startalign
\NC φ(x+y) \NC = φ(x_1 e_1 + x_2 e_2 + x_3 e_3 + y_1 e_1 + y_2 e_2 + y_3 e_3) \NR
\NC \NC = φ((x_1 + y_1) e_1 + (x_2 + y_2) e_2 + (x_3 + y_3) e_3) \NR
\NC \NC = (2(x_1 + y_1) - (x_2 + y_2))e_1 + \NR
\NC \NC \qquad + ((x_1 + y_1) - 2(x_2 + y_2))e_2 + (x_2 + y_2 - x_3 - y_3)e_3 \NR
\NC \NC = (2x_1 - x_2)e_1 + (x_1 - 2x_2)e_2 + (x_2 - x_3)e_3 + \NR
\NC \NC \qquad + (2y_1 - y_2)e_1 + (y_1 - 2y_2)e_2 + (y_2 - y_3)e_3 \NR
\NC \NC = φ(x_1 e_1 + x_2 e_2 + x_3 e_3) + φ(y_1 e_1 + y_2 e_2 + y_3 e_3) \NR
\NC \NC = φ(x) + φ(y) \NR
\stopalign \stopformula
Проверим условие однородности:
\startformula \startalign
\NC φ(αx) \NC = φ(α x_1 e_1 + α x_2 e_2 + α x_3 e_3) \NR
\NC \NC = (2αx_1 − αx_2)e_1 + (αx_1 − 2αx_2)e_2 + (αx_2 − αx_3)e_3 \NR
\NC \NC = α((2x_1 - x_2)e_1 + (x_1 - 2x_2)e_2 + (x_2 - x_3)e_3) \NR
\NC \NC = α φ(x_1 e_1 + x_2 e_2 + x_3 e_3) \NR
\NC \NC = α φ(x) \NR
\stopalign \stopformula
Поскольку условия аддитивности и однородности выполняются для $φ$, то $φ$ --- линейный отображение.
При этом, $φ$ отображает трёхмерное пространство в себя, т.е. $φ$ --- линейный оператор.
Найдём матрицу данного линейного оператора:
\startformula
M = \pmat{φ(e_1); φ(e_2); φ(e_3)} = \pmat{2, 1, 0; -1, -2, 1; 0, 0, -1}
\stopformula
Найдём ядро данного линейного оператора:
\startformula
XM = 0
\stopformula
\startformula \startcases
2x_1 - x_2 + 0 = 0 \NR
 x_1 -2x_2 + 0 = 0 \NR
 0  +  x_2 - x_3 = 0 \NR
\stopcases \Rightarrow \quad \startcases
x_1 = 0 \NR
x_2 = 0 \NR
x_3 = 0 \NR
\stopcases \Rightarrow \quad
\text{Ker}\,φ = \left{(0,0,0)\right}
\stopformula
Зная ядро, найдём ранг и дефект оператора $φ$:
\startformula \text{def}\,φ = \text{dim}\left(\text{Ker}\,φ\right) = 0 \stopformula
\startformula \text{rank}\,φ = 3 - \text{def}\,φ = 3 - 0 = 3\stopformula


\subject {Задание №2}
Покажите, что отображение $φ$, состоящее в умножении всех матриц из $M(2, {\Bbb R})$ слева на матрицу $A$,
будет линейным оператором $M(2, {\Bbb R})$. Найдите матрицу этого линейного оператора в базисе
$E_1$, $E_2$, $E_3$, $E_4$. Вычислите Ker\,$φ$.
\startformula
	E_1 = \pmat{1,0;0,0} \quad E_2=\pmat{0,1;0,0} \quad E_3=\pmat{0,0;1,0} \quad E_4=\pmat{0,0;0,1}
	\quad A=\pmat{1,2;-3,1}
\stopformula
\subsubject {Решение}
Проверим условие аддитивности:
\startformula
φ(X+Y)	= A(X + Y) = AX + AY = φ(X) + φ(Y)
\stopformula
Проверим условие однородности:
\startformula
φ(αX)	= φ(α X) = AαX = αAX = αφ(X)
\stopformula
Поскольку условия аддитивности и однородности выполняются для $φ$, то $φ$ --- линейный отображение.
При этом, $φ$ отображает четырёхмерное пространство в себя, т.е. $φ$ --- линейный оператор.
Найдём координаты векторов, которые получаются при приминении 
Найдём разложение по векторам базиса $E_1$, $E_2$, $E_3$, $E_4$.
векторов $φ(E_1)$, $φ(E_2)$, $φ(E_3)$, $φ(E_4)$.
\startformula φ(E_1) = A E_1 = \pmat{1,2;-3,1} \cdot \pmat{1,0;0,0} = \pmat{1,0;-3,0} = E_1 - 3E_3 \stopformula
\startformula φ(E_2) = A E_2 = \pmat{1,2;-3,1} \cdot \pmat{0,1;0,0} = \pmat{0,1;0,-3} = E_2 - 3E_4 \stopformula
\startformula φ(E_3) = A E_3 = \pmat{1,2;-3,1} \cdot \pmat{0,0;1,0} = \pmat{2,0;1,0} = 2E_1 + E_3 \stopformula
\startformula φ(E_4) = A E_4 = \pmat{1,2;-3,1} \cdot \pmat{0,0;0,1} = \pmat{0,2;0,1} = 2E_2 + E_4 \stopformula
Найдём матрицу данного этого линейного оператора в базисе $E_1$, $E_2$, $E_3$, $E_4$.
\startformula
	M_φ = \pmat{φ(E_1); φ(E_2); φ(E_3); φ(E_4)} = \pmat{1,0,-3,0;0,1,0,-3;2,0,1,0;0,2,0,4}
\stopformula
Найдём ядро данного линейного оператора:
\startformula \startcases
 x_1 + 0 - 3x_3 + 0 = 0 \NR
 0 + x_2 + 0 - 3x_4 = 0 \NR
 2x_1 + 0 + x_3 + 0 = 0 \NR
 0 + 2x_2 + 0 + x_4 = 0 \NR
\stopcases \Rightarrow \quad \startcases
x_1 = 0 \NR
x_2 = 0 \NR
x_3 = 0 \NR
x_4 = 0 \NR
\stopcases \Rightarrow \quad
\text{Ker}\,φ = \left{(0,0,0,0)\right}
\stopformula


\subject {Задание №3}
Пусть φ --- отображение пространства ${\Bbb R}_3[x]$ в себя,
которое многочлену $f(x)$ ставит в соответствие многочлен
\startformula g(x) = f(x + a) − f(x − b),\ \text{где}\ a = 2,\ b = 1 \stopformula
Покажите, что $φ$ --- линейный оператор ${\Bbb R}_3[x]$.
Найдите матрицу этого оператора в базисе $1$, $x$, $x^2$, $x^3$.
\subsubject {Решение}
Проверим условие аддитивности:
\startformula \startalign
\NC φ(f_1 + f_2)
	\NC = (f_1+f_2)(x+2) - (f_1+f_2)(x-1) \NR
\NC	\NC = f_1(x+2) - f_1(x-1) + f_2(x+2) - f_2(x-1) = φ(f_1) + φ(f_2) \NR
\stopalign \stopformula
Проверим условие однородности:
\startformula
φ(αf)	= αf(x+2) - αf(x-1) = α(f(x+2) - f(x-1)) = αφ(f)
\stopformula
Поскольку условия аддитивности и однородности выполняются для $φ$, то $φ$ --- линейный оператор.
Найдём разложение по векторам базиса $1$, $x$, $x^2$, $x^3$.
векторов $φ(1)$, $φ(x)$, $φ(x^2)$, $φ(x^3)$.
\startformula φ(1) = 1 - 1 = 0 \stopformula
\startformula φ(x) = (x + 2) - (x - 1) = 1 \stopformula
\startformula φ(x^2) = (x + 2)^2 - (x - 1)^2 = 3+6x \stopformula
\startformula φ(x^3) = (x + 2)^3 - (x - 1)^3 = 9 + 9x + 9x^2 \stopformula
Составим матрицу данного этого линейного оператора в базисе $1$, $x$, $x^2$, $x^3$.
\startformula
	M_φ = \pmat{φ(1); φ(x); φ(x^2); φ(x^3)} = \pmat{0,0,0,0;1,0,0,0;3,6,0,0;9,9,9,0}
\stopformula


\subject {Задание №4}
Составьте матрицу линейного оператора $φ$ в базисе $e_1$, $e_2$, $e_3$ действительного линейного пространства $V$,
если $φ$ векторы
\startformula x_1 = ae_1 − e_2 + e_3,\ x_2 = e_1 + be_2 − e_3,\ x_3 = e_1 − e_2 − ae_3 \stopformula
переводит соответственно в векторы
\startformula 
	y_1 = e_1 + 2e_2 + be_3,\ y_2 = ae_2 + e_3,\ y_3 = e_1 + be_3,\ \text{где}\ a = 2,\ b = 1
\stopformula
Найдите образ вектора $z = e_1 + 2e_2 + 3e_3$ под действием оператора $φ$.
\subsubject {Решение}
Найдём координаты векторов $φ(e_1)$, $φ(e_2)$, $φ(e_3)$.
\startformula \startcases
φ(x_1) = y_1 \NR
φ(x_2) = y_2 \NR
φ(x_3) = y_3 \NR
\stopcases \Rightarrow \quad \startcases
φ(2e_1 - e_2 + e_3) = e_1 + 2e_2 + e_3 \NR
φ(e_1 + e_2 - e_3) = 2e_2 + e_3 \NR
φ(e_1 - e_2 - 2e_3) = e_1 + e_3 \NR
\stopcases \Rightarrow \stopformula
\startformula \startcases
2φ(e_1) - φ(e_2) + φ(e_3) = e_1 + 2e_2 + e_3 \NR
φ(e_1) + φ(e_2) - φ(e_3) = 2e_2 + e_3 \NR
φ(e_1) - φ(e_2) - 2φ(e_3) = e_1 + e_3 \NR
\stopcases \Rightarrow \quad \startcases
%[[1/3, 4/3, 2/3 ], [ -4/9, 8/9, 1/9 ], [ -1/9, 2/9, -2/9 ] ]
φ(e_1) =  \frac13 e_1 + \frac43 e_2 + \frac23 e_3 \NR
φ(e_2) = -\frac49 e_1 + \frac89 e_2 + \frac19 e_3 \NR
φ(e_3) = -\frac19 e_1 + \frac29 e_2  -\frac29 e_3 \NR
\stopcases \stopformula
Составим матрицу данного этого линейного оператора в базисе $e_1$, $e_2$, $e_3$, $e_4$.
\startformula
M_φ = \pmat{φ(e_1); φ(e_2); φ(e_3); φ(e_4)} = \pmat{
	\frac13, \frac43, \frac23;
	-\frac49, \frac89, \frac19;
	-\frac19, \frac29, -\frac29
}
\stopformula
Найдём образ вектора $z = e_1 + 2e_2 + 3e_3$ под действием оператора $φ$.
\startformula
φ(z) = z M_φ = \pmat{1,2,3}\cdot
	\pmat{\frac13, \frac43, \frac23; -\frac49, \frac89, \frac19; -\frac19, \frac29, -\frac29}
	% [ -8/9, 34/9, 2/9 ]
	= \pmat{ -\frac89, \frac{34}{9}, \frac29 }
\stopformula


\subject {Задание №5}
В базисе $e_1$, $e_2$, $e_3$ линейный оператор $φ$ имеет матрицу $B$.
Как изменится матрица оператора $φ$ при переходе к базису $e′_1$, $e′_2$, $e′_3$?
\startformula
	e′_1 = 2e_1 + 1e_2\qquad e′_2 = −e_1 + 2e_2 + 3e_3\qquad e′_3 = e_1 + e_2 + e_3
\stopformula \startformula
	B = \pmat{0,1,-3;2,4,1;0,3,-3}
\stopformula
\subsubject {Решение}
Для нахождения матрицы $A$ линейного оператора $φ$ в базисе $e'_1$, $e'_2$, $e'_3$
воспользуемся тем фактом, что $A=T\cdot B\cdot T^{-1}$, 
где $T$ --- матрица перехода об базиса $e_1$, $e_2$, $e_3$ к базису $e′_1$, $e′_2$, $e′_3$.
\startformula
% [[2,1,0],[-1,2,3],[1,1,1]] * [[0,1,-3],[2,4,1],[0,3,-3]] * [[2,1,0], [-1,2,3], [1,1,1]]^(-1);
% [ [ 37/2, 15/2, -55/2 ], [ 36, 16, -52 ], [ 45/2, 19/2, -67/2 ] ]
A = TBT^{-1} = \pmat{2,1,0;-1,2,3;1,1,1} \pmat{0,1,-3;2,4,1;0,3,-3} \pmat{2,1,0;-1,2,3;1,1,1}^{-1}
	= \pmat{\frac{37}2, \frac{15}2, -\frac{55}2; 36, 16, -52; \frac{45}2, \frac{19}2, -\frac{67}2}
\stopformula


\subject {Задание №6}
В трехмерном линейном пространстве с базисом $e_1$, $e_2$, $e_3$ заданы векторы
\startformula
	u_1 = e_1 + 2e_2 + 3e_3\qquad u_2 = e_1 + 3e_2 + 4e_3\qquad u_3 = e_1 + 3e_2 + 5e_3
\stopformula \startformula
	v_1 = 3e_1 + 2e_2 + 2e_3\qquad v_2 = e_1 + e_2 + e_3\qquad v_3 = 4e_1 + 3e_2 + 4e_3
\stopformula
Докажите, что векторы $u_1$, $u_2$, $u_3$ и $v_1$, $v_2$, $v_3$ образуют базисы данного пространства.
В базисе $u_1$, $u_2$, $u_3$ линейный оператор $φ$ задан матрицей $B$.
Вычислите матрицу оператора $φ$ в базисе $v_1$, $v_2$, $v_3$.
\startformula
	B = \pmat{0,1,-3;2,4,1;0,3,-3}
\stopformula
\subsubject {Решение}
Для того, чтобы доказать, что векторы $u_1$, $u_2$, $u_3$ образуют базис трёхмерного линейного пространства,
проверим, что определитель матрицы, составленной из данных векторов, не равен нулю.
\startformula
% [[1,2,3],[1,3,4],[1,3,5]];
\dmat{u_1; u_2; u_3} = \dmat{1, 2, 3; 1, 3, 4; 1, 3, 5} = 1
\stopformula
Для того, чтобы доказать, что векторы $v_1$, $v_2$, $v_3$ образуют базис трёхмерного линейного пространства,
проверим, что определитель матрицы, составленной из данных векторов, не равен нулю.
\startformula
% [[3, 2, 2],[1, 1, 1],[4, 3, 4]];
\dmat{v_1; v_2; v_3} = \dmat{3, 2, 2; 1, 1, 1; 4, 3, 4} = 1
\stopformula
% v := [[3,1,3],[2,1,3],[2,1,4]]; u:=[[1,1,1],[2,3,3],[3,4,5]];
Найдём матрицу $T$ перехода от базиса $u_1$, $u_2$, $u_3$ к базису $v_1$, $v_2$, $v_3$.
\startformula
T = V\cdot U^{-1} = \pmat{3, 1, 3; 2, 1, 3; 2, 1, 4} \pmat{1, 1, 1; 2, 3, 3; 3, 4, 5}^{-1}
	= \pmat{5,-4,2; 2,-3,2; 1,-4,3}
\stopformula
Вычислим матрицу $A$ линейного оператора $φ$ в базисе $v_1$, $v_2$, $v_3$.
\startformula
%[[5,-4,2],[2,-3,2],[1,-4,3]]*[[0,1,-3],[2,4,1],[0,3,-3]]*[[5,-4,2],[2,-3,2],[1,-4,3]]^(-1);
A = T\cdot B\cdot T^{-1} = \pmat{5,-4,2; 2,-3,2; 1,-4,3} \pmat{0,1,-3;2,4,1;0,3,-3}\pmat{5,-4,2; 2,-3,2; 1,-4,3}^{-1}
	= \pmat{153, -497, 221; 97, -316, 131; 112, -366, 164}
\stopformula


\subject {Задание №7}
В линейном пространстве ${\Bbb R}^3$ линейный оператор $φ$ задан матрицей $C$.
Найдите базисы ядра и образа, ранг и дефект оператора $φ$.
\startformula
	C = \pmat{4,-8,-4;0,4,2;1,-2,-1}
\stopformula
\subsubject {Решение}
Найдём ядро линейного оператора $φ$.
\startformula
xC = 0 \quad \Rightarrow \quad \startcases
 4x_1 + 0   +  x_3 = 0 \NR
-8x_1 + 4x_2 -2x_3 = 0 \NR
-4x_1 + 2x_2 - x_3 = 0 \NR
\stopcases \Rightarrow \quad \startcases
x_2 = 0 \NR
x_3 = -4x_1 \NR
\stopcases \stopformula
\startformula
\text{Ker}\,φ = \left{(α,0,-4α)\ |\ a\in{\Bbb R}\right}
\stopformula
Базис ядра данного линейного оператора:
\startformula
	e = \pmat{1, 0, -4}
\stopformula
Найдём образ линейного оператора $φ$.
\startformula
φ(e_1) = 4e_1 - 8e_2 - 4e_3
\qquad φ(e_2) = 4e_2 + 2e_3
\qquad φ(e_3) = e_1 - 2e_2 - e_3
\stopformula
\startformula
\dmat{4,-8,-4;0,4,2;1,-2,-1} = 0 \qquad \qquad \dmat{4,-8;0,4} = 16 \neq 0
\stopformula
Базис образа данного линейного оператора:
\startformula
	e'_1 = \pmat{4, -8, -4} \qquad \qquad e'_2 = \pmat{0, 4, 2}
\stopformula
Найдём ранг и дефект оператора $φ$.
\startformula \text{rank}\,φ = \text{dim}\,\text{Im}\,φ = 2 \stopformula
\startformula \text{def}\,φ = \text{dim}\,\text{Ker}\,φ = 1 \stopformula

\subject {Задание №8}
В некотором базисе пространства ${\Bbb R}^3$ операторы $f$ и $g$ имеют матрицы $B$ и $C$.
Найдите матрицы операторов $f + g$, $5f$, $fg$. Обратимы ли операторы $f$ и $g$?
\startformula
	B = \pmat{0,1,-3;2,4,1;0,3,-3} \qquad \qquad C = \pmat{4,-8,-4;0,4,2;1,-2,-1}
\stopformula
\subsubject {Решение}
Пусть, $M_f = B$, $M_g = C$. Найдём матрицы операторов $f + g$, $5f$, $fg$, восопльзовавшись
тем фактом, что $M_{f+g} = M_f + M_g,\ M_{5f} = 5 M_f,\ M_{fg} = M_g \cdot M_f$.
% f := [[0,1,-3],[2,4,1],[0,3,-3]]; g := [[4,-8,-4], [0,4,2], [1,-2,-1]];
\startformula
M_{f+g} = M_f + M_g
	= \pmat{0,1,-3;2,4,1;0,3,-3} + \pmat{4,-8,-4;0,4,2;1,-2,-1}
	= \pmat{4, -7, -7;2, 8, 3; 1, 1, -4}
\stopformula
\startformula
M_{5f} = 5M_f
	= 5 \cdot \pmat{0,1,-3;2,4,1;0,3,-3}
	= \pmat{0, 5, -15; 10, 20, 5; 0, 15, -15} 
\stopformula
\startformula
M_{fg} = M_g \cdot M_f
	= \pmat{4,-8,-4;0,4,2;1,-2,-1} \cdot \pmat{0,1,-3;2,4,1;0,3,-3}
	= \pmat{-16, -40, -8; 8, 22, -2; -4, -10, -2}
\stopformula
Для того, чтобы определить обратимы ли операторы $f$ и $g$, найдём определители соответствующих им матриц:
\startformula \left|M_f\right| = \dmat{0,1,-3;2,4,1;0,3,-3} = -12 \stopformula
\startformula \left|M_g\right| = \dmat{4,-8,-4;0,4,2;1,-2,-1} = 0 \stopformula
Таким образом, оператор $f$ обратим, а $g$ не обратим.

\stoptext
