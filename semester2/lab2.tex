\definecolor[headingcolor][r=1]
\definecolor[solutionheadercolor][r=0.5, b=1]
\setuphead[subject][color=headingcolor, indentnext=yes]
\setuphead[subsubject][color=solutionheadercolor, indentnext=yes]
\setuphead[subject][style={\ss\bfa}, before={\bigskip\bigskip\bigskip\bigskip}, after={}]
\setuphead[subsubject][style={\ss\bfa}, before={}, after={}]

\setuppapersize[A4]
\setuplayout[backspace=2cm, topspace=1cm, header=1cm, bottomspace=1cm,
    footer=1cm, width=middle, height=middle]
\setupbodyfont[libertinus, 14pt]
\mainlanguage[ru]
\setupwhitespace[medium]
\setupindenting[medium, yes]

\definemathmatrix[lmatrix][simplecommand=lmat, left={\left\{\,}, right={\,\right. }]
\definemathmatrix[umatrix][simplecommand=umat, left={\left[\,}, right={\,\right. }]
\definemathmatrix[pmatrix][simplecommand=pmat, left={\left(\,}, right={\,\right) }]
\definemathmatrix[dmatrix][simplecommand=dmat, left={\left|\,}, right={\,\right| }]
\definemathmatrix[gmatrix][simplecommand=grid, left={\left.\,}, right={\,\right. }]
\definemathcommand[arctg][nolop]{\mfunction{arctg}}

\setupindenting[medium,yes]
%\setupformulas[align=right]
\starttext
\setuppagenumbering[state=stop]
\centerline { Гомельский государственный университет имени Франциска Скорины }
\centerline { факультет математики и технологий программирования }
\vfill \vfill
\centerline { Лабораторная работа №2 }
\centerline { «Базис и размерность линейного пространства» }
\vfill \vfill
{\leftskip 0.55\hsize \noindent
  Выполнил:\\Хамков Владислав (ПИ-11)\\
  Проверил:\\Васильев Александр Федорович}
\vfill
\centerline { Май 2022 }
\page

\setuppagenumbering[state=start]
\setuppagenumber[number=1]
\setuppagenumbering[location={footer,center}]
\setupfootertexts[\quad \date \hfill Страница \pagenumber\ / \lastpagenumber \quad]

\subject {Задание №1}
Докажите, что в линейном пространстве ${\Bbb R}_n[x]$ многочлены
$1$, $x−a$, $(x−a)^2$..., $(x−a)^n$, $a ∈ \Bbb{R}$, образуют базис.
Найдите координаты многочлена $f(x)$ в этом базисе.
\startformula
	a=-1 \qquad\qquad f(x)=x^5 + 4x^4 + 4x^3 − 8x^2 − 32x − 32
\stopformula
\subsubject {Решение}
Линейное пространство ${\Bbb R}_n[x]$ имеет размерность $n+1$, т.е. совпадает с количеством многочленов.
Поэтому достаточно проверить, что многочлены $1$, $x−a$, $(x−a)^2$..., $(x−a)^n$, $a ∈ \Bbb{R}$, линейно независимы.
Пусть
\startformula
	α_0 + α_1(x-a) + α_2(x-a)^2 + ... + α_n(x-a)^n ≡ 0
\stopformula
Проведя замену $t=x-a$, получаем
\startformula
	α_0 + α_1 t + α_2 t^2 + ... + α_n t^n ≡ 0
\stopformula
Полученное уравнение выполняестся для всех $t$ при условии, что
\startformula
	α_0 = α_1 = α_2 = ... = α_n = 0
\stopformula
Это означает, что многочлены $1$, $x−a$, $(x−a)^2$..., $(x−a)^n$, $a ∈ \Bbb{R}$ линейно независимы в общем случае.
В частности, они --- линейно независимы при $a=-1$.
Следовательно, они образуют базис в линейном пространстве ${\Bbb R}_n[x]$.
\par
Найдём координаты многочлена $f(x)$ в этом базисе.
\startformula \startalign
\NC f(x) \NC =x^5 + 4x^4 + 4x^3 - 8x^2 - 32x - 32 \NR
\NC \NC = (x+1)^5-(x+1)^4-2(x+1)^3-6(x+1)^2-15(x+1)-9 \NR
\NC \NC = -9 + (-15)(x+1) + (-6)(x+1)^2 +(-2)(x+1)^3 + (-1)(x+1)^4 + (x+1)^5 \NR
\stopalign \stopformula
Таким образом, координатной строкой $f(x)$ в базисе $1$, $x+1$, $(x+1)^2$..., $(x+1)^n$ является
$\left(-9, -15, -56, -2, -1, 1, 0, 0, 0, ...\right)$.


\subject {Задание №2}
\par
Докажите, что система векторов $e1$, $e2$, $e3$ образует базис в линейном
пространстве ${\Bbb R}^3$, и найдите координаты вектора $a$ в этом базисе.
\startformula
	e_1=(2,1,-3) \qquad e_2=(3,2,-5) \qquad e_3=(1,-1,1) \qquad a=(6,2,-7)
\stopformula
\subsubject {Решение}
Линейное пространство ${\Bbb R}^3$ имеет размерность три, т.е. совпадает с количеством даных нам векторов.
Поэтому достаточно проверить, что векторы $e_1$, $e_2$, $e_3$ линейно независимы.
Пусть
\startformula
	α_1 (2,1,-3) + α_2 (3,2,-5) + α_3 (1,-1,1) = (0, 0, 0)
\stopformula
Переходим к системе
\startformula
	\lmat{ 2α_1 + 3α_2 + α_3 = 0; α_1 + 2α_2 - α_3 = 0; -3α_1 -5α_2 +α_3 = 0}
\stopformula
Решаяa эту систему, получим $α_1 = α_2 = α_3 = 0$, т.е. векторы $e_1$, $e_2$, $e_3$ линейно независимы.
Следовательно, они образуют базис в линейном пространстве ${\Bbb R}^3$.
Найдём координаты $x$, $y$, $z$ вектора $a$ в этом базисе.
Пусть
\startformula
	x (2,1,-3) + y (3,2,-5) + z (1,-1,1) = (6, 2, -7)
\stopformula
Из системы
\startformula
	\lmat{ 2x + 3y + z = 6; x + 2y - z = 2; -3x -5y +z = -7}
\stopformula
получим $x = y = z = 1$


\subject {Задание №3}
Докажите, что комплексные числа $z_1$ и $z_2$ образуют базис действительного
линейного пространства комплексных чисел $\Bbb{C}$,
и запишите координатную строку числа $−5 + 4i$ в этом базисе.
\startformula
	z_1 = 2i \qquad\qquad z_2 = 3 + 2i
\stopformula
\subsubject {Решение}
Линейное пространство ${\Bbb C}$ имеет размерность два, т.е. совпадает с количеством даных нам комплексных чисел.
Поэтому достаточно проверить, что комплексные числа $z_1$ и $z_2$ линейно независимы.
Пусть
\startformula
	α_1 2i + α_2 (3+2i) = 0
\stopformula
Переходим к системе
\startformula
	\lmat{ 0 + 3α_2 = 0; 2α_1 + 2α_2 = 0 }
\stopformula
Решая эту систему, получим $α_1 = α_2 = 0$, т.е. числа $z_1$ и $z_2$ линейно независимы.
Следовательно, они образуют базис в линейном пространстве ${\Bbb C}$.
Найдём координаты вектора $-5+4i$ в этом базисе.
Пусть
\startformula
	x 2i + y (3+2i) = -5 + 4i
\stopformula
Из системы
\startformula
	\lmat{ 0 + 3y = -5; 2x + 2y = 4 }
\stopformula
получаем $x = \frac{11}3, y=-\frac53$.
Т.е. координатной строкой числа $-5+4i$ в базисе $(z_1, z_2)$ является $\left(\frac{11}3, -\frac53\right)$.



\subject {Задание №4}
Докажите, что матрицы $E_1$, $E_2$, $E_3$, $E_4$ образуют базис линейного
пространства $M(2, {\Bbb R})$, и запишите разложение по векторам этого базиса
вектора $A$.
\startformula
	E_1=\pmat{1,2;3,2} \quad E_2=\pmat{2,3;4,3} \quad E_3=\pmat{3,4;6,1} \quad E_4=\pmat{-1,4;7,9}
	\quad A=\pmat{-1,1;2,-1}
\stopformula
\subsubject {Решение}
Линейное пространство $M(2, {\Bbb R})$ имеет размерность четыре, т.е. совпадает с количеством даных нам матриц.
Поэтому достаточно проверить, что матрицы $E_1$, $E_2$, $E_3$, $E_4$ линейно независимы.
Пусть
\startformula
	a_1 \pmat{1,2;3,2} + α_2\pmat{2,3;4,3} + α_3\pmat{3,4;6,1} +α_4\pmat{-1,4;7,9} = \pmat{0, 0; 0, 0}
\stopformula
Переходим к системе
\startformula
	\lmat{  α_1 + 2 α2 + 3 α_3  -  α_4 = 0;
		2 α_1 + 3 α2 + 4 α_3 + 4 α_4 = 0;
		3 α_1 + 4 α2 + 6 α_3 + 7 α_4 = 0;
		2 α_1 + 3 α2 + 1 α_3 + 9 α_4 = 0  }
\stopformula
Решая эту систему, получим $α_1 = α_2 = α_3 = α_4 = 0$, т.е. векторы $E_1$, $E_2$, $E_3$, $E_4$ линейно независимы.
Следовательно, они образуют базис в линейном пространстве $M(2, {\Bbb R})$.
Найдём координаты $x$, $y$, $z$, $w$ вектора $A$ в этом базисе.
Пусть
\startformula
	x\,\pmat{1,2;3,2} + y\,\pmat{2,3;4,3} + z\,\pmat{3,4;6,1} + w\,\pmat{-1,4;7,9} = \pmat{-1, 1; 2, -1}
\stopformula
Из системы
\startformula
	\lmat{    x + 2 y + 3 z  -  w = -1;
		2 x + 3 y + 4 z + 4 w = 1;
		3 x + 4 y + 6 z + 7 w = 2;
		2 x + 3 y + 1 z + 9 w = -1  }
\stopformula
получим $x = -41$, $y = 9$, $z = 9$, $w=5$. Т.е.
\startformula
	A = -41 E_1 + 9 E_2 + 9 E_3 + 5 E_4
\stopformula


\subject {Задание №5}
Базис пространства решений однородной системы линейных уравнений называется
фундаментальной системой решений. Найдите размерность и укажите фундаментальную
систему решений пространства решений следующей однородной системы линейных уравнений:
\startformula
	\startcases
		\NC x_1 + 4x_2 + x_3 + 2x_4 = 0 \NR
		\NC 2x_1 + 7x_2 - 2x_3 + 0 = 0 \NR
		\NC -x_1 + 3x_2 + 0 - x_4 = 0 \NR
   	\stopcases
\stopformula
\subsubject {Решение}
Найдём решения данной системы линейных уравнений.
\startformula
	\startcases
		\NC x_1 + 4x_2 + x_3 + 2x_4 = 0 \NR
		\NC 2x_1 + 7x_2 - 2x_3 + 0 = 0 \NR
		\NC -x_1 + 3x_2 + 0 - x_4 = 0 \NR
   	\stopcases
	\Longrightarrow \quad
	\startcases
		\NC x_1 + x_4 = 0 \NR
		\NC x_2 = 0 \NR
		\NC x_3 + x_4 = 0 \NR
   	\stopcases
	\Longrightarrow \quad
	\startcases
		\NC x_1 = -x_4 \NR
		\NC x_2 = 0 \NR
		\NC x_3 = -x_4 \NR
   	\stopcases
\stopformula
Таким образом мы получаем пространство решений:
\startformula
	V = \left{(-λ, 0, -γ, γ)\ |\ γ \in {\Bbb R}\right}
\stopformula
Известно, что пространство решений любой однородной системы линейных уравнений образует действительное
линейное пространство относительно операций сложения строк и умножения числа на строку.
В частности, $V$ является действительным линейным пространством.
Базисом линейного пространства $V$ можно выбрать вектор
\startformula e = (-1, 0, -1, 1) \stopformula
Размерность линейного пространства $V$ равна единице.



\subject {Задание №6}
Как изменится матрица перехода от одного базиса к другому в результате следующего преобразования:
поменять местами два последних вектора первого базиса?
\subsubject {Решение}
Очевидно, что поменяются местами два последних стобца матрицы перехода.
Например, пусть
\startformula \pmat{e'_1; e'_2; e'_3} = \pmat{
		a_{11}, a_{12}, a_{13};
		a_{21}, a_{22}, a_{23};
		a_{31}, a_{32}, a_{33} } \pmat{e_1; e_2; e_3}
\stopformula
Тогда, меняя местами два последних вектора базиса $e_1$, $e_2$, $e_3$, получаем следующее уравнение:
\startformula \pmat{e'_1; e'_2; e'_3} = \pmat{
		a_{11}, a_{13}, a_{12};
		a_{21}, a_{23}, a_{22};
		a_{31}, a_{33}, a_{32} } \pmat{e_1; e_3; e_2}
\stopformula


\subject {Задание №7}
Даны векторы $e_1$, $e_2$, $e_3$, $a_1$, $a_2$, $a_3$ линейного пространства ${\Bbb R}^3$.
Докажите, что векторы $e_1$, $e_2$, $e_3$ и $a_1$, $a_2$, $a_3$ образуют базисы линейного пространства ${\Bbb R}^3$.
Найдите матрицу перехода от базиса $e_1$, $e_2$, $e_3$ к базису $a_1$, $a_2$, $a_3$.
Найдите матрицу перехода от базиса $a_1$, $a_2$, $a_3$ к базису $e_1$, $e_2$, $e_3$.
Найдите координаты вектора $c = (2, 0, 1)$ в базисе $e_1$, $e_2$, $e_3$.
Найдите координаты вектора $x = e_1 − e_2 + 2e_3$ в базисе $a_1$, $a_2$, $a_3$.
\startformula
	\startgmatrix[distance=2cm]
		\NC e_1=(1,2,1) \NC e_2=(2,3,3) \NC e_3=(3,8,2) \NR
		\NC a_1=(1,0,2) \NC a_2=(3,-1,4) \NC a_3=(2,-2,1) \NR
	\stopgmatrix
\stopformula
\subsubject {Решение}
Докажем, что векторы $e_1$, $e_2$, $e_3$ образуют базисы линейного пространства ${\Bbb R}^3$.
Линейное пространство ${\Bbb R}^3$ имеет размерность три, т.е. совпадает с количеством даных нам векторов.
Поэтому достаточно проверить, что векторы $e_1$, $e_2$, $e_3$ линейно независимы.
Пусть
\startformula α_1 (1,2,1) + α_2 (2,3,3) + α_3 (3,8,2) = (0, 0, 0) \stopformula
Переходим к системе
\startformula \lmat{ α_1 + 2α_2 + 3α_3 = 0; 2α_1 + 3α_2 + 8α_3 = 0; 1α_1 + 3α_2 + 2 α_3 = 0} \stopformula
Решаяa эту систему, получим $α_1 = α_2 = α_3 = 0$, т.е. векторы $e_1$, $e_2$, $e_3$ линейно независимы.
Следовательно, они образуют базис в линейном пространстве ${\Bbb R}^3$.
\par 
Докажем, что векторы $a_1$, $a_2$, $a_3$ образуют базисы линейного пространства ${Bbb R}^3$.
Линейное пространство ${\Bbb R}^3$ имеет размерность три, т.е. совпадает с количеством даных нам векторов.
Поэтому достаточно проверить, что векторы $a_1$, $a_2$, $a_3$ линейно независимы.
Пусть
\startformula α_1 (1,0,2) + α_2 (3,-1,4) + α_3 (2,-2,1) = (0, 0, 0) \stopformula
Переходим к системе
\startformula \lmat{ α_1 + 3α_2 + 2α_3 = 0; -1α_2 - 2α_3 = 0; 2α_1 + 4α_2 + α_3 = 0} \stopformula
Решаяa эту систему, получим $α_1 = α_2 = α_3 = 0$, т.е. векторы $a_1$, $a_2$, $a_3$ линейно независимы.
Следовательно, они образуют базис в линейном пространстве ${\Bbb R}^3$.
Найдём матрицу перехода от базиса $e_1$, $e_2$, $e_3$ к базису $a_1$, $a_2$, $a_3$.
\startformula \startalign
\NC \pmat{a_1; a_2; a_3} \NC = A\,\pmat{e_1; e_2; e_3} \NR
\NC \pmat{1,0,2;3,-1,4;2,-2,1}\pmat{\hat{i}; \hat{j}; \hat{k}}
	\NC = A\,\pmat{1,2,1;2,3,3;3,8,2}\pmat{\hat{i}; \hat{j}; \hat{k}} \NR
\NC \pmat{1,0,2;3,-1,4;2,-2,1} \NC = A\,\pmat{1,2,1;2,3,3;3,8,2} \NR
\NC \pmat{1,0,2;3,-1,4;2,-2,1}\pmat{1,2,1;2,3,3;3,8,2}^{-1} \NC = A \NR
\stopalign \stopformula
\startformula
	A = \pmat{1,0,2;3,-1,4;2,-2,1} \pmat{1,2,1;2,3,3;3,8,2}^{-1}
	= \quad \pmat{4, 0, -1; 31, -5, -6; 39, -8, -7}
\stopformula
Найдём матрицу перехода от базиса $a_1$, $a_2$, $a_3$ к базису $e_1$, $e_2$, $e_3$.
\startformula \startalign
\NC \pmat{e_1; e_2; e_3} \NC = B\,\pmat{a_1; a_2; a_3} \NR
\NC \pmat{1,2,1;2,3,3;3,8,2}\pmat{\hat{i}; \hat{j}; \hat{k}}
	\NC = B\,\pmat{1,0,2;3,-1,4;2,-2,1}\pmat{\hat{i}; \hat{j}; \hat{k}} \NR
\NC \pmat{1,2,1;2,3,3;3,8,2} \NC = B\,\pmat{1,0,2;3,-1,4;2,-2,1} \NR
\NC \pmat{1,2,1;2,3,3;3,8,2} \pmat{1,0,2;3,-1,4;2,-2,1}^{-1} \NC = B \NR
\stopalign \stopformula
\startformula
	B = \pmat{1,2,1;2,3,3;3,8,2} \pmat{1,0,2;3,-1,4;2,-2,1}^{-1}
	= \quad \pmat{-13,8,-5; -17,11,-7; -53,32,-20}
\stopformula
Найдём координаты вектора $c = (2, 0, 1)$ в базисе $e_1$, $e_2$, $e_3$.
\startformula \startalign
\NC c \NC = (2,0,1) = 2\hat{i} + 0 + \hat{k} = \pmat{2, 0, 1} \pmat{\hat{i}; \hat{j}; \hat{k}} \NR
\NC \NC = \pmat{2, 0, 1} \pmat{1,2,1;2,3,3;3,8,2}^{-1} \pmat{e_1; e_2; e_3} = \pmat{29, -6, -5} \pmat{e_1; e_2; e_3} \NR
\NC \NC = 29e_1 -6e_2 -5e_3 \NR
\stopalign \stopformula
Найдём координаты вектора $x = e_1 − e_2 + 2e_3$ в базисе $a_1$, $a_2$, $a_3$.
\startformula \startalign
\NC x \NC = e_1 - e_2 + 2e_3 = \pmat{1, -1, 2} \pmat{e_1; e_2; e_3} \NR
\NC \NC = \pmat{1, -1, 2} B \pmat{a_1; a_2; a_3} = \pmat{-102, 61, -38} \pmat{a_1; a_2; a_3} \NR
\NC \NC = -102a_1 + 61a_2 - 38a_3 \NR
\stopalign \stopformula


\subject {Задание №8}
Пусть $(−3, 2, 0)$ --- координатная строка вектора $a$ в базисе $e'_1$, $e'_2$, $e'_3$ 
действительного линейного пространства $V$,
$A$ --- матрица перехода от базиса $e'_1$, $e'_2$ , $e'_3$ пространства $V$ к базису $e_1$, $e_2$, $e_3$.
Найдите координаты вектора $a$ в базисе $e_1$, $e_2$, $e_3$.
\startformula
	A = \pmat{-1,2,3;2,0,1;-2,2,2}
\stopformula
\subsubject {Решение}
Найдём координаты вектора $a = -3e_1 + 2e_2 + 0$ в базисе $e'_1$, $e'_2$, $e'_3$.
\startformula \startalign
\NC a \NC = -3e_1 + 2e_2 + 0 = \pmat{-3, 2, 0} \pmat{e_1; e_2; e_3} \NR
\NC \NC = \pmat{-3, 2, 0} A^{-1}\,\pmat{e'_1; e'_2; e'_3} = \NR
\NC \NC = \pmat{-3, 2, 0} \pmat{-1,2,3;2,0,1;-2,2,2}^{-1}\pmat{e'_1; e'_2; e'_3} = \NR
\NC \NC = \pmat{-3, 2, 0} \pmat{-1,1,1;-3,2,\frac72;2,-1,-2}\pmat{e'_1; e'_2; e'_3} = \NR
\NC \NC = \pmat{-3, 1, 4} \pmat{e'_1; e'_2; e'_3} = \NR
\NC \NC = -3a_1 + 1a_2 - 4a_3 \NR
\stopalign \stopformula


\stoptext
