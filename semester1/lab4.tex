\setuppapersize[A4]
\setuplayout[backspace=2cm, topspace=1cm, header=1cm, bottomspace=1cm,
    footer=1cm, width=middle, height=middle]
\setupbodyfont[libertinus, 14pt]
\mainlanguage[ru]
\setuppagenumbering[location={footer,center}]

\setupformulas[align=right]
\starttext
\setuppagenumbering[state=stop]
\centerline { Гомельский государственный университет имени Франциска Скорины }
\centerline { факультет математики и технологий программирования }
\vfill \vfill
\centerline { Лабораторная работа №4 }
\centerline { «Сравнения» }
\vfill \vfill
{\leftskip 0.55\hsize \noindent
  Выполнил:\\Хамков Владислав (ПИ-11)\\
  Проверил:\\Васильев Александр Федорович}
\vfill
\centerline { Ноябрь 2021 }
\page

\setuppagenumbering[state=start]
\setuppagenumber[number=1]
\centerline {Задание №1}
\startformula \startalign[align=left]
\NC a = 178^{274},\ b = 22\quad rem(a, b) =\ ? \NR
\NC 178 \equiv 2 \pmod{22} \NR
\NC 178^{11} \equiv 2^{11} = 2048 \equiv 2 \pmod{22} \NR
\NC 178^{21} \equiv 2^{21} = 2^{11} \cdot 2^{10} \equiv 2 \cdot 12 \equiv 2 \pmod{22} \NR
\NC 178^{1 + 10n} \equiv 2 \pmod{22} \NR
\NC 178^{274} = 178^{271 + 3} = 178^{271} \cdot 178^3 \equiv 2^{271} \cdot 2^3 \equiv 2 \cdot 8
    \equiv 16 \pmod{22} \NR
\stopalign \stopformula
\vfil


\centerline {Задание №2}
\startformula \startalign[align=left]
\NC a = 178^{274},\ b = 10\quad rem(a, b) =\ ? \NR
\NC 178 \equiv 8 \pmod{10} \NR
\NC 178^5 \equiv 8^5 = 32768 \equiv 8 \pmod{10} \NR
\NC 178^9 \equiv 8^9 = 8^5 \cdot 8^4 \equiv 8 \cdot 6 \equiv 8 \pmod{10} \NR
\NC 178^{4n + 1} \equiv 8 \pmod{10} \NR
\NC 178^{274} = 178^{273} \cdot 178 \equiv 8^{273} \cdot 8 \equiv 8 \cdot 8 \equiv 4 \pmod{10} \NR
\stopalign \stopformula
\vfil


\centerline {Задание №3}
\startformula \startalign[align=left]
\NC c = 4^{323} + 26,\ d = 15;\quad c \equiv 0\ ?\pmod{d} \NR
\NC 4^2 = 16 \equiv 1 \pmod{15} \NR
\NC c = 4^{323} + 26 = 4^{322} \cdot 4 + 11 + 15 \equiv 1 \cdot 4 + 11 = 15 \equiv 0 \pmod{15} \NR
\stopalign \stopformula
\vfil


\centerline {Задание №4}
\startformula \startalign[align=left]
\NC a = 523908;\quad φ(a) =\ ? \NR
\NC a = 523908 = 2^2 \cdot 3^5 \cdot 7^2 \cdot 11 \NR
\NC φ(a) = (2^2 - 2)(3^5 - 3^4)(7^2 - 7)(11 - 1)
    = 2 \cdot 162 \cdot 42 \cdot 10 = 136080 \NR
\stopalign \stopformula
\vfil


\centerline {Задание №5}
\startformula \startalign[align=left]
\NC m = 7,\quad \mathbb{Z}_m\ ? \NR
\NC НОД(7, 2) = НОД(7, 3) = НОД(7, 4) = НОД(7, 5) = НОД(7, 6) = 1 \NR
\NC 2 \cdot 4 \equiv 3 \cdot 5 \equiv 6 \cdot 6 \equiv 1 \pmod{7} \NR
\stopalign \stopformula
Таким образом, обратимыми элементами являются $1, 2, 3, 4, 5, 6$.
Для них обратными элементами являются $1, 4, 5, 2, 3, 6$ соответственно.
\page


\centerline {Задание №6}
\startformula \startalign[align=left]
\NC m = 6,\quad \mathbb{Z}_m\ ? \NR
\NC НОД(6, 2) = НОД(6, 4) = 2; НОД(6, 3) = 3; НОД(6, 5) = 1 \NR
\stopalign \stopformula
Таким образом, обратимыми элементами являются $1$ и $5$.
Таблица умножения для них выглядит следующим образом:
\starttable[|1|1|1|]
\HL
\VL * \VL 1 \VL 5 \VL\AR
\HL
\VL 1 \VL 1 \VL 5 \VL\AR
\HL
\VL 5 \VL 5 \VL 1 \VL\AR
\HL
\stoptable
\vfil


\centerline {Задание №7}
\startformula \startalign[align=left]
\NC 6x \equiv 22 \pmod{13} \NR
\NC 6x \equiv 9 \pmod{13} \NR
\NC 11 \cdot 6x \equiv 11 \cdot 9 \pmod{13} \NR
\NC 66x \equiv 99 \pmod{13} \NR
\NC x \equiv 8 \pmod{13} \NR
\stopalign \stopformula
Ответ: $x = 13k + 8,\quad \forall k \in \mathbb{Z}$
\vfil


\centerline {Задание №8}
\startformula \startalign[align=left]
\NC 15x - 21y = 42 \NR
\NC 5(3x - 4y) - y = 5\cdot8 + 2 \NR
\NC -y \equiv 2 \pmod{5} \Rightarrow y = 5k + 3 \forall k \in \mathbb{Z} \NR
\NC 5(3x - 4(5k + 3)) - 5k - 3 = 40 + 2 \NR
\NC 5(3x - 20k - 12) = 45 + 5k \NR
\NC 3x - 20k - 12 = 9 + k \NR
\NC 3x = 21 + 21k \NR
\NC x = 7 + 7k \NR
\stopalign \stopformula
Ответ: $\forall k \in \mathbb{Z}\quad x = 7 + 7k,\ y = 5k + 3$
\stoptext
