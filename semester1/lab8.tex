\setuppapersize[A4]
\setuplayout[backspace=2cm, topspace=1cm, header=1cm, bottomspace=1cm,
    footer=1cm, width=middle, height=middle]
\setupbodyfont[libertinus, 14pt]
\mainlanguage[ru]
\setuppagenumbering[location={footer,center}]
\definemathmatrix[lmatrix][simplecommand=lmat, left={\left\{\,}, right={\,\right. }]
\definemathmatrix[umatrix][simplecommand=umat, left={\left[\,}, right={\,\right. }]
\definemathmatrix[pmatrix][simplecommand=pmat, left={\left(\,}, right={\,\right) }]
\definemathmatrix[dmatrix][simplecommand=dmat, left={\left|\,}, right={\,\right| }]

%\setupformulas[align=right]
\starttext
\setuppagenumbering[state=stop]
\centerline { Гомельский государственный университет имени Франциска Скорины }
\centerline { факультет математики и технологий программирования }
\vfill \vfill
\centerline { Лабораторная работа №8 }
\centerline { «Системы линейных уравнений» }
\vfill \vfill
{\leftskip 0.55\hsize \noindent
  Выполнил:\\Хамков Владислав (ПИ-11)\\
  Проверил:\\Васильев Александр Федорович}
\vfill
\centerline { Декабрь 2021 }
\page

\setuppagenumbering[state=start]
\setuppagenumber[number=1]
\centerline {Задание №1}
\startformula
\lmat{3x_1 - x_2 + 2x_3 = -2; 4x_1 + 3x_3 = -1; x_1 + 3x_2 + x_3 = 3} \rightarrow
\pmat{3,-1,2,-2; 4,0,3,-1; 1,3,1,3} \rightarrow
\pmat{1,3,1,3;3,-1,2,-2; 4,0,3,-1} \rightarrow
\stopformula
\startformula
\rightarrow
\pmat{1,3,1,3; 0,-10,-1,-11; 0,-12,-1,-13} \rightarrow
\pmat{1,3,1,3; 0,-10,-1,-11; 0,0,\frac15,\frac15} \rightarrow
\lmat{x_1 + 3x_2 + x_3 = 3;  10x_2+x_3=11;  x_3=1}
\stopformula
Ранг матрицы --- 3, а ранг расширенной матрицы -- 3.\\
Поэтому система совместна.
Решим её методом Гауса:
\startformula
\lmat{x_1 + 3x_2 + x_3 = 3;  10x_2+x_3=11;  x_3=1} \rightarrow
\pmat{1,3,1,3; 0,10,1,11; 0,0,1,1} \rightarrow
\lmat{x_1 = 3 - x_3 - 3x_2;  x_2=\frac{11-x_3}{10};  x_3=1} \rightarrow
\lmat{x_1 = -1;  x_2=1;  x_3=1}
\stopformula
Решим систему методом Крамера:
\startformula
\Delta = \dmat{1,3,1; 0,10,1; 0,0,1} = 10 \quad\quad
\Delta_1 = \dmat{3,3,1; 11,10,1; 1,0,1} = -10 \quad
\stopformula
\startformula
\Delta_2 = \dmat{1,3,1; 0,11,1; 0,1,1} = 10 \quad\quad
\Delta_3 = \dmat{1,3,3; 0,10,11; 0,0,1} = 10 \quad
\stopformula
\startformula
x_1 = \frac{\Delta_1}{\Delta} = \frac{-10}{10} = -1 \quad\quad
x_2 = \frac{\Delta_2}{\Delta} = \frac{10}{10} = 1 \quad\quad
x_3 = \frac{\Delta_3}{\Delta} = \frac{10}{10} = 1 \quad\quad
\stopformula
Решим систему матричным методом:
\startformula
X = \pmat{1,3,1; 0,10,1; 0,0,1}^{-1} \pmat{3;11;1}
= \pmat{1,-\frac3{10}, -\frac7{10}; 0, \frac1{10}, -\frac1{10}; 0,0,1} \pmat{3;11;1}
= \pmat{1,-\frac3{10}, -\frac7{10}; 0, \frac1{10}, -\frac1{10}; 0,0,1} \pmat{3;11;1}
\stopformula
\startformula
= \pmat{3 - \frac{33}{10} - \frac{7}{10};  \frac{11}{10} - \frac1{10};  1}
= \pmat{-1;1;1} \rightarrow
\lmat{x_1 = -1;  x_2=1;  x_3=1}
\stopformula
\page

\centerline {Задание №2}
\startformula
\lmat{
	-3x_1 +  x_2 -  (1-i)x_3  = 2;
	 2x_1 - 4x_2 +     i x_3  = 3-i;
	 -x_1 - 3x_2 - (1-2i)x_3 = 4
} \rightarrow
\pmat{
	-3, 1, i-1, 2;
	2, -4, i, 3-i;
	-1, -3, 2i-1, 4
}
\stopformula
\startformula
\dmat{
	-3, 1, i-1;
	2, -4, i;
	-1, -3, 2i-1
} = 0
\quad\quad
\dmat{
	1, i-1, 2;
	-4, i, 3-i;
	-3, 2i-1, 4
} = -1 - 9i
\stopformula
Ранг матрицы меньше ранга расширенной матрицы, поэтому система несовместна.
\startformula
\lmat{
	 ix_1 +    2 x_2   -3 x_3  = -2-4i;
	 2x_1 + (i-1)x_2 +  i x_3  = 2-i;
	  x_1     -i x_2 +  4 x_3 = 1+5i
} \rightarrow
\pmat{
	i, 2, -3, -2-4i;
	2, i-1, i, 2-i;
	1, -i, 4, 1+5i
} \rightarrow
\stopformula
\startformula
\rightarrow
\pmat{
	1, -i, 4, 1+5i;
	i, 2, -3, -2-4i;
	2, i-1, i, 2-i
} \rightarrow
\pmat{
	1, -i, 4, 1+5i;
	0, 1, -3-4i, 3-5i;
	0, -3+3i, -2+9i, -6-i
} \rightarrow
\stopformula
\startformula
\rightarrow
\pmat{
	1, -i, 4, 1+5i;
	0, 1, -3-4i, 3-5i;
	0, 0, -23+6i, -12-25i
} \rightarrow
\lmat{
	x_1 = \frac{441}{565} - \frac{278}{565}i;
	x_2 = -\frac{103}{113} - \frac{76}{113}i;
	x_3 = \frac{126}{565} + \frac{647}{565}i
}
\stopformula
\vfil

\centerline {Задание №3}
\startformula
\lmat{
	3 x_1 + 4 x_2 + 3 x_3  -4 x_4 = 11;
	4 x_1 + 5 x_2   - x_3  -2 x_4 = 7;
	- x_1   - x_2 + 4 x_3  -2 x_4 = 4
} \rightarrow
\lmat{
	  x_1   + x_2  -4 x_3 + 2 x_4 = -4;
	3 x_1 + 4 x_2 + 3 x_3  -4 x_4 = 11
} \rightarrow
\stopformula
\startformula
\lmat{
	  x_1   + x_2  -4 x_3 + 2 x_4 = -4;
                  x_2 + 15 x_3  -10 x_4 = 23
} \rightarrow
\lmat{
	x_1 = -27 + 19 x_3 - 12 x_4;
        x_2 = 23 - 15 x_3 - 10 x_4
} \rightarrow
\stopformula
\vfil

\centerline {Задание №4}
\startformula
\lmat{
	6x_1 + 3x_2 - 5x_3 = 2;
	9x_1 + 4x_2 - 7x_3 = 3;
	3x_1 + x_2 - 2x_3 = 1
} \rightarrow
\lmat{
	3x_1 + x_2 - 2x_3 = 1;
	9x_1 + 4x_2 - 7x_3 = 3
} \rightarrow
\stopformula
\startformula
\lmat{
	3x_1 + x_2 - 2x_3 = 1;
	       x_2 - x_3 = 0
} \rightarrow
\lmat{
	x_1 = \frac13 (1 + x_3);
	x_2 = x_3
}
\stopformula
\page

\centerline {Задание №5}
\startformula
\lmat{
	x_1 + 7x_2 + x_3 = 1;
	x_1 + αx_2 + x_3 = -1;
	6x_1 + x_2 + αx_3 = 1
} \rightarrow
\pmat{
	1, 7, 1, 1;
	1, α, 1, -1;
	6, 1, α, 1
} \rightarrow
\stopformula
\startformula
\pmat{
	1, 7, 1, 1;
	0, α-7, 0, -2;
	0, -41, α-6, -5
} \rightarrow
\dmat{
	1, 7, 1;
	0, α-7, 0;
	0, -41, α-6
} = (α-7)(α-6)
\stopformula
Если $α \ne 7$ и $α \ne 6$, то ранг матрицы = 3 \Rightarrow система совместна. \\
Если $α = 7$, то во второй строчке получаем $0=-2$, т.е. система несовместна. \\
Если $α = 6$, то вторая и третья строки противоречат друг другу, т.е. система несовместна.
\vfil

\centerline {Задание №6}
\startformula
A = \pmat{ -3, 1; 1, -2 }
\quad\quad\quad
AB = BA
\quad\quad\quad
B = \pmat{ x_1, x_2; x_3, x_4 }
\stopformula
\startformula
\pmat{ -3, 1; 1, -2 }\pmat{ x_1, x_2; x_3, x_4 }
= \pmat{ x_1, x_2; x_3, x_4 } \pmat{ -3, 1; 1, -2 }
\stopformula
\startformula
\pmat {
	x_3 - 3x_1,  x_4 - 3x_2;
	x_1 - 2x_3,  x_2 - 2x_4
} =
\pmat {
	x_2 - 3x_1,  x_1 - 2x_2;
	x_4 - 3x_3,  x_3 - 2x_4
}
\stopformula
\startformula
\lmat {
	x_3 - x_2 = 0;
	x_4 - x_1 - x_2 = 0;
	x_1 + x_3 - x_4 = 0;
	x_2 - x_3 = 0
} \rightarrow
\lmat {
	x_1 = x_4 - x_3;
	x_2 = x_3
} \rightarrow
B = \pmat{a - b, b; b, a}
\stopformula
\vfil

\stoptext
