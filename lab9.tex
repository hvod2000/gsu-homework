\:setuppapersize[A4]
\setuplayout[backspace=2cm, topspace=1cm, header=1cm, bottomspace=1cm,
    footer=1cm, width=middle, height=middle]
\setupbodyfont[libertinus, 14pt]
\mainlanguage[ru]
\setuppagenumbering[location={footer,center}]
\definemathmatrix[lmatrix][simplecommand=lmat, left={\left\{\,}, right={\,\right. }]
\definemathmatrix[umatrix][simplecommand=umat, left={\left[\,}, right={\,\right. }]
\definemathmatrix[pmatrix][simplecommand=pmat, left={\left(\,}, right={\,\right) }]
\definemathmatrix[dmatrix][simplecommand=dmat, left={\left|\,}, right={\,\right| }]
\definemathcommand[arctg][nolop]{\mfunction{arctg}}

%\setupformulas[align=right]
\starttext
\setuppagenumbering[state=stop]
\centerline { Гомельский государственный университет имени Франциска Скорины }
\centerline { факультет математики и технологий программирования }
\vfill \vfill
\centerline { Лабораторная работа №9 }
\centerline { «Многочлены» }
\vfill \vfill
{\leftskip 0.55\hsize \noindent
  Выполнил:\\Хамков Владислав (ПИ-11)\\
  Проверил:\\Васильев Александр Федорович}
\vfill
\centerline { Декабрь 2021 }
\page

\setuppagenumbering[state=start]
\setuppagenumber[number=1]
\centerline {Задание №1}
\startformula
f(x) = \bar{1}x^5 + \bar{1}x^3 + \bar{3}x + \bar{2} \quad\quad\quad g(x) = \bar{1}x^3 + \bar{1}x
\stopformula
\startformula
f(x) = (\bar{1}x^3 + \bar{1}x) \cdot (\bar{1}x^2) + (\bar{3}x + \bar{2})
\stopformula
\vfil

\centerline {Задание №2}
\startformula
f(x) = 2x^4 + 3x^3 - 3x^2 - 5x + 2 \quad\quad\quad g(x) = 2x^3 + x^2 - x - 1
\stopformula
\startformula
f = g \cdot (x + 1) + (-3x^2 - 3x + 3)
\stopformula
\startformula
g = (-3x^2 - 3x + 3) \cdot (\frac13 - \frac23x) + (2x - 2)
\stopformula
\startformula
(-3x^2 - 3x + 3) = (2x - 2) \cdot(-3 -\frac32x) + (-3)
\stopformula
\startformula
НОД\left(f, g\right) = -3 = \left(-3x^2 - 3x + 3\right) - \left(2x - 2\right)\left(-3 - \frac32x\right) =
\stopformula
\startformula
= \left(-3x^2-3x+3\right) - \left(g - \left(-3x^2-3x+3\right)\left(\frac13-\frac23x\right)\right)\left(-3-\frac32x\right) =
\stopformula
\startformula
= f - g\left(x+1\right) - \left(g - \left(f-g\left(x+1\right)\right)\left(\frac13-\frac23x\right)\right)\left(-3-\frac32x\right) =
\stopformula
\startformula
= g\cdot\left(3 - x^3 - \frac52 x^2\right) + f\left(x^2 +\frac32 x\right)
\stopformula
\vfil

\centerline {Задание №3}
\startformula
f(x) = 5x^4 + 2ix^3 + 5x - i
\quad \quad \quad
x_0 = 1 + i
\stopformula
\midaligned{
\bTABLE[width=2cm]
\bTR\bTD         \eTD\bTD $5$ \eTD\bTD $2i$ \eTD\bTD $0$ \eTD\bTD $5$ \eTD\bTD $-i$ \eTD\eTR
\bTR\bTD $1 + i$ \eTD\bTD $5$ \eTD\bTD $5+7i$ \eTD\bTD $-2+12i$ \eTD\bTD $-9+10i$ \eTD\bTD $-19$ \eTD\eTR
\eTABLE
}
\startformula
f(x_0) = (5x^3 + (5+7i)x^2 + (-2+12i)x - 9+10i) \cdot (x - x_0) - 19 = 0 - 19 = -19
\stopformula
\page

\centerline {Задание №4}
\startformula
f(x) = 2x^4 + 3x^3 - 3x^2 - 5x + 2 \quad\quad\quad x + 1
\stopformula
\midaligned{
\bTABLE[width=2cm]
\bTR\bTD         \eTD\bTD $2$ \eTD\bTD $ 3$ \eTD\bTD $-3$ \eTD\bTD $-5$ \eTD\bTD $2$ \eTD\eTR
\bTR\bTD $-1$    \eTD\bTD $2$ \eTD\bTD $ 1$ \eTD\bTD $-4$ \eTD\bTD $-1$ \eTD\bTD $3$ \eTD\eTR
\bTR\bTD $-1$    \eTD\bTD $2$ \eTD\bTD $-1$ \eTD\bTD $-3$ \eTD\bTD $ 2$ \eTD\bTD \eTD\eTR
\bTR\bTD $-1$    \eTD\bTD $2$ \eTD\bTD $-3$ \eTD\bTD $ 0$ \eTD\bTD \eTD\bTD \eTD\eTR
\bTR\bTD $-1$    \eTD\bTD $2$ \eTD\bTD $-5$ \eTD\bTD \eTD\bTD \eTD\bTD \eTD\eTR
\bTR\bTD $-1$    \eTD\bTD $2$ \eTD\bTD \eTD\bTD \eTD\bTD \eTD\eTR
\eTABLE
}
\startformula
f(x) = 2(x+1)^4 - 5(x+1)^3 + 2(x+1) + 3
\stopformula
\vfil

\centerline {Задание №5}
\startformula
f(x) = 4ax^3 - 2bx^2 + 3cx - 4
\stopformula
\midaligned{
\bTABLE[width=2cm]
\bTR\bTD     \eTD\bTD $4a$ \eTD\bTD $-2b$ \eTD\bTD $3c$ \eTD\bTD $-4$ \eTD\eTR
\bTR\bTD $1$ \eTD\bTD $4a$ \eTD\bTD $4a-2b$ \eTD\bTD $4a-2b+3c$ \eTD\bTD $4a-2b+3c-4$ \eTD\eTR
\bTR\bTD $1$ \eTD\bTD $4a$ \eTD\bTD $8a-2b$ \eTD\bTD $12a-4b+3c$ \eTD\bTD \eTD\eTR
\bTR\bTD $1$ \eTD\bTD $4a$ \eTD\bTD $12a-2b$ \eTD\bTD \eTD\bTD \eTD\eTR
\eTABLE
}
\startformula
\lmat{
	4a - 2b + 3c - 4 = 0;
	12a - 4b + 3c = 0;
	12a - 2b = 0
} \rightarrow
\lmat{
	a = 1;
	b = 6;
	c = 4
} 
\stopformula
\vfil

\centerline {Задание №6}
\startformula
f(x) = -3x^4 + 5x^3 + 8x^2 - 20x + 16
\stopformula
\midaligned{
\bTABLE[width=2cm]
\bTR\bTD     \eTD\bTD $-3$ \eTD\bTD $5$ \eTD\bTD $8$ \eTD\bTD $-20$ \eTD\bTD $16$ \eTD\eTR
\bTR\bTD  $1$ \eTD\bTD $-3$ \eTD\bTD $2$ \eTD\bTD $10$ \eTD\bTD $-10$ \eTD\bTD $6$ \eTD\eTR
\bTR\bTD  $2$\eTD\bTD $-3$ \eTD\bTD $-1$ \eTD\bTD $6$ \eTD\bTD $-8$ \eTD\bTD $0$ \eTD\eTR
\bTR\bTD  $2$\eTD\bTD $-3$ \eTD\bTD $-7$ \eTD\bTD $-8$ \eTD\bTD $-24$ \eTD\bTD  \eTD\eTR
\bTR\bTD $-2$\eTD\bTD $-3$ \eTD\bTD $5$  \eTD\bTD $-4$ \eTD\bTD $0$ \eTD\bTD  \eTD\eTR
\eTABLE
}
\startformula
f(x) = (x+2)(x-2)(-3x^2 + 5x - 4x)
\stopformula
\midaligned{
Корни: $-2,\ 2,\ \frac56+\frac16\sqrt{23}i,\ \frac56-\frac16\sqrt{23}i $
}
\page

\centerline {Задание №7}
\startformula
f(x) = x^4 + 2x^3 + 10x^2 - 6x + 65
\quad\quad\quad x_1 = -2 + 3i
\stopformula
\startformula
f(x) = x^4 + 2x^3 + 10x^2 - 6x + 65 = (x+2-3i)(x+2+3i)(x^2 - 2x + 5) =
\stopformula
\startformula
= (x+2-3i)(x+2+3i)(x-1+2i)(x-1-2i)
\stopformula
\midaligned{
Корни: $-2+3i,\ -2-3i,\ 1+2i,\ 1-2i$
}
\vfil

\centerline {Задание №8}
\startformula
f(x) = x^5 + x^4 + 5x^3 + 5x^2 - 6x - 6
\stopformula
\midaligned{
\bTABLE[width=2cm]
\bTR\bTD      \eTD\bTD $1$ \eTD\bTD $1$ \eTD\bTD $5$  \eTD\bTD $5$  \eTD\bTD $-6$ \eTD\bTD $-6$ \eTD\eTR
\bTR\bTD $1$  \eTD\bTD $1$ \eTD\bTD $2$ \eTD\bTD $7$  \eTD\bTD $12$ \eTD\bTD $6$  \eTD\bTD $ 0$ \eTD\eTR
\bTR\bTD $1$  \eTD\bTD $1$ \eTD\bTD $3$ \eTD\bTD $10$ \eTD\bTD $22$ \eTD\bTD $28$ \eTD\bTD\eTD\eTR
\bTR\bTD $-1$ \eTD\bTD $1$ \eTD\bTD $1$ \eTD\bTD $6$  \eTD\bTD $6$  \eTD\bTD $0$  \eTD\bTD\eTD\eTR
\bTR\bTD $-1$ \eTD\bTD $1$ \eTD\bTD $0$ \eTD\bTD $6$  \eTD\bTD $0$  \eTD\bTD\eTD\bTD\eTD\eTR
\eTABLE
}
\startformula
f(x) = (x+1)(x+1)(x-1)(x^2 + 6)\ \text{в}\ \mathbb{R}
\stopformula
\startformula
f(x) = (x+1)(x+1)(x-1)(x-\sqrt{6}i)(x+\sqrt6i)\ \text{в}\ \mathbb{C}
\stopformula
\\
\startformula
g(x) = x^4 - x^2 + 9\ \text{в}\ \mathbb{R}
\stopformula
\startformula
g(x) = x^4 - x^2 + 9
= \left(x^2 - \frac{1+\sqrt{1 - 4\cdot6}}2\right)  \left(x^2 - \frac{1-\sqrt{1 - 4\cdot6}}2\right) =
\stopformula
\startformula
= \left(x^2 - \frac{1+i\sqrt{35}}2\right)  \left(x^2 - \frac{1-i\sqrt{35}}2\right) =
\stopformula
\startformula
= 
\left(x^2 - 3\left(\cos{\arctg{\sqrt{35}}} + i\sin{\arctg{\sqrt{35}}}\right)\right)
\left(x^2 - 3\left(\cos{\arctg{-\sqrt{35}}} + i\sin{\arctg{-\sqrt{35}}}\right)\right)
=
\stopformula
\startformula
= 
\left(x - \sqrt3\left(\cos{\frac{\arctg{\left(\sqrt{35}\right)}}2} + i\sin{\frac{\arctg{\left(\sqrt{35}\right)}}2}\right)\right)
\stopformula \startformula
\left(x - \sqrt3\left(\cos{\left(\pi + \frac{\arctg{\left(\sqrt{35}\right)}}2\right)} + i\sin{\left(\pi + \frac{\arctg{\left(\sqrt{35}\right)}}2\right)}\right)\right)
\stopformula \startformula
\left(x - \sqrt3\left(\cos{\frac{\arctg{\left(-\sqrt{35}\right)}}2} + i\sin{\frac{\arctg{\left(-\sqrt{35}\right)}}2}\right)\right)
\stopformula \startformula
\left(x - \sqrt3\left(\cos{\left(\pi + \frac{\arctg{\left(-\sqrt{35}\right)}}2\right)} + i\sin{\left(\pi + \frac{\arctg{\left(-\sqrt{35}\right)}}2\right)}\right)\right)
\stopformula
\vfil
П.С. Почему последнее задание такое страшное?

\stoptext
